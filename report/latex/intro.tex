Computed Tomography (CT) and Magnetic Resonance Imaging (MRI) are widely utilised imaging techniques, used in medical diagnostics. CT scans utilise X-rays to create detailed images, and are particularly effective at visualising bones and internal injuries. These scans provide images of bony structures and are often used in cases including trauma due to the speed at which they can be taken.

MRI scans, on the other hand, use powerful superconducting electromagnets which can produce detailed images of soft tissues such as the brain, which we will be looking at in this paper. MRI is particularly effective for brain scans because it can differentiate between these soft tissues based on their varying water content and relaxation properties. In recent years, segmentation of MRI scans has become very accurate, and can be used to diagnose many health issues. However, MRI faces a few problems; namely that it is time consuming and expensive compared to other types of scan. As such, it is desirable for cheaper and quicker scans such as those made using CT to be automatically registered to an MRI, to leverage the strengths of both imaging methods. The purpose of the project is exactly that. If successful, this would allow such scans to take place in situations where time is of the essence and help medical personnel in taking more informed decisions.

There are many methods for registering images, with a commonly used one being the Elastix framework\cite{elastix}, which uses iterative methods. These methods are quite slow however, and deep learning-based methods such as Synthmorph have been proposed which drastically reduces registration time down to less than a minute on CPU and just a few seconds on GPU\cite{synthmorph}. Generally speaking, two different main types of registration exist and are currently used in the field, depending on the use cases, rigid and non-rigid. Rigid registration methods transform images based on constrained affine transformation matrix. The matrix is constrained such that it can only perform a subset of affine transformations, rotation and translation. This type of registration, despite being the ``weakest'' in terms of how much they can transform is nevertheless sufficient in many medical contexts.

Non-rigid registration includes both deformable and affine transformations. With deformable registration methods, each voxel can be individually transformed, effectively allowing transformations of inputs into completely unrecognisable outputs. Affine registration methods are essentially an extension of the rigid methods. While the affine transformation matrix used in rigid registration is constrained, it is when performing affine registration. This opens up the possibility to both scale, shear and reflect inputs, in addition to the aforementioned rigid capabilities.

Synthmorph consists of several different registration models, including both deformable, rigid and affine models. In this paper, we will be looking at only the rigid and affine versions of the models for registration of intra-subject pairs of CT and MRI scans.

All code used in this paper can be found on \href{https://github.com/Cavtheman/ct_mri_reg/}{Github}\footnote{\href{https://github.com/Cavtheman/ct_mri_reg/}{https://github.com/Cavtheman/ct\_mri\_reg/}}, though it has not as of the time of writing been made into a state usable for others.
