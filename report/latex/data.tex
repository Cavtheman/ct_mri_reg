For this project, the SynthRAD dataset\cite{synthradData} will serve as the primary resource. The SynthRAD dataset is a substantial collection of brain and pelvis CT and T1 MRI scans. Only the brain scans are used in this project. The data is also been completely anonymised, with all patient data removed.

The SynthRAD dataset is useful to this work, since it has been thoroughly preprocessed and registered using conventional (rigid) iterative methods. The methods by which this has been done can be read in the corresponding literature. This thorough preprocessing of the data is crucial, as it allows us in this paper to use the registered images as a ground truth for ``perfect'' registration later in this paper. As such, we will use the SynthRAD dataset for constructing the training, validation, and test sets used throughout this thesis.

It is noted in the original paper\cite{synthradData} that they have concentrated outliers from the norm in this training dataset, but have not provided their test set for public use, due to the fact that it was used in the SynthRAD2023 Grand Challenge. As such, these outliers may mean that the results shown in this paper are not entirely representative. However, this may have proven beneficial for training purposes, as the model should be less susceptible to these types of outliers.
