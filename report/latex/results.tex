To demonstrate the improvements made here over the capabilities of original Synthmorph model, we have first evaluated its performance on all four test datasets without any fine-tuning. However, for purposes of making the results more medically plausible, most of the models have not been evaluated on the test sets containing images that have been sheared, and metrics have not been calculated for them on the test sets. It will be clearly denoted where the shear test sets have been used.

% Baseline synthmorph
As is to be expected, Synthmorph without any tuning performs very poorly when trying to register CT to MRI scans. This can be seen in Tables \ref{table:aff_ct_mr_baseline} and \ref{table:rigid_ct_mr_baseline} where both the $\|A\hat{B} - I\|$ and $\|B - \hat{B}\|$ loss values are roughly 2-5 times larger than the corresponding loss values of MRI-MRI registration in Tables \ref{table:aff_mr_mr_baseline} and \ref{table:rigid_mr_mr_baseline}.

%%%%%%%%%%%%%%%%%%%%%%%%%%%%%%%%%%%%%%%%%%%%%%%%%%%%%%%%%%%%%%%%%%%%%%%%%%%%%%%%
% Updated affine SynthMorph results
\begin{table}[h!]
\centering
\begin{minipage}{0.48\textwidth}
\centering
\resizebox{\textwidth}{!}{%
\begin{tabular}{c|cc}\hline
Generation Parameters & $\|A\hat{B} - I\|$ & $\|B - \hat{B}\|$ \\\hline
$\theta \pm 0.4$, $T \pm 40$, No Shear                 & 7.4554  & 7.4554  \\
$\theta \pm 0.4$, $T \pm 40$, \text{Shear} $= \pm 0.1$ & 7.5588  & 7.5306  \\
$\theta \pm 0.2$, $T \pm 20$, No Shear                 & 3.8292  & 3.8292  \\
$\theta \pm 0.2$, $T \pm 20$, \text{Shear} $= \pm 0.1$ & 4.4211  & 4.4242  \\\hline
\end{tabular}%
}
\caption{MRI-MRI registration results with different transformations using the untuned affine SynthMorph model.}
\label{table:aff_mr_mr_baseline}
\end{minipage}
\hfill
\begin{minipage}{0.48\textwidth}
\centering
\resizebox{\textwidth}{!}{%
\begin{tabular}{c|cc}\hline
Generation Parameters & $\|A\hat{B} - I\|$ & $\|B - \hat{B}\|$ \\\hline
$\theta \pm 0.4$, $T \pm 40$, No Shear                 & 44.1560 & 44.1560 \\
$\theta \pm 0.4$, $T \pm 40$, \text{Shear} $= \pm 0.1$ & 44.5351 & 44.1592 \\
$\theta \pm 0.2$, $T \pm 20$, No Shear                 & 42.0576 & 42.0576 \\
$\theta \pm 0.2$, $T \pm 20$, \text{Shear} $= \pm 0.1$ & 43.3569 & 42.9668 \\\hline
\end{tabular}%
}
\caption{CT-MRI registration results with different transformations using the untuned affine SynthMorph model.}
\label{table:aff_ct_mr_baseline}
\end{minipage}
\end{table}

% Updated rigid SynthMorph results
\begin{table}[h!]
\centering
\begin{minipage}{0.48\textwidth}
\centering
\resizebox{\textwidth}{!}{%
\begin{tabular}{c|cc}\hline
Generation Parameters & $\|A\hat{B} - I\|$ & $\|B - \hat{B}\|$ \\\hline
$\theta \pm 0.4$, $T \pm 40$, No Shear                 & 8.8768  & 8.8768  \\
$\theta \pm 0.4$, $T \pm 40$, \text{Shear} $= \pm 0.1$ & 19.1687 & 19.1985 \\
$\theta \pm 0.2$, $T \pm 20$, No Shear                 & 4.5827  & 4.5827  \\
$\theta \pm 0.2$, $T \pm 20$, \text{Shear} $= \pm 0.1$ & 16.8906 & 16.8533 \\\hline
\end{tabular}%
}
\caption{MRI-MRI registration results with different transformations using the untuned rigid SynthMorph model.}
\label{table:rigid_mr_mr_baseline}
\end{minipage}
\hfill
\begin{minipage}{0.48\textwidth}
\centering
\resizebox{\textwidth}{!}{%
\begin{tabular}{c|cc}\hline
Generation Parameters & $\|A\hat{B} - I\|$ & $\|B - \hat{B}\|$ \\\hline
$\theta \pm 0.4$, $T \pm 40$, No Shear                 & 24.9288 & 24.9288 \\
$\theta \pm 0.4$, $T \pm 40$, \text{Shear} $= \pm 0.1$ & 31.9698 & 32.7524 \\
$\theta \pm 0.2$, $T \pm 20$, No Shear                 & 23.8434 & 23.8434 \\
$\theta \pm 0.2$, $T \pm 20$, \text{Shear} $= \pm 0.1$ & 29.9778 & 30.7856 \\\hline
\end{tabular}%
}
\caption{CT-MRI registration results with different transformations using the untuned rigid SynthMorph model.}
\label{table:rigid_ct_mr_baseline}
\end{minipage}
\end{table}
 % text inside
After training both the affine and rigids versions of the Synthmorph model on our augmented datasets, we find that the network starts outperforming the baseline model. For MRI-MRI registration this can easily be explained via slight overfitting to the dataset at hand. This can be seen in Tables \ref{table:mr_mr_affine_tuned_results} and \ref{table:mr_mr_rigid_tuned_results}.

For the CT-MRI results we also see significant improvements thanks to the training. The $\|A\hat{B} - I\|$ and $\|B - \hat{B}\|$ loss values can be seen in Tables \ref{table:ct_mr_affine_tuned_results} and \ref{table:ct_mr_rigid_tuned_results} to have improved to a point where the CT-MRI registrations perform roughly equivalently to the MRI-MRI registration.

% Tuned with $\|A\hat{B} - I\|$ loss
\begin{table}[h!]
\centering
\begin{minipage}{0.48\textwidth}
\centering
\resizebox{\textwidth}{!}{%
\begin{tabular}{c|cc}\hline
Generation Parameters & $\|A\hat{B} - I\|$ & $\|B - \hat{B}\|$ \\\hline
$\theta \pm 0.4$, $T \pm 40$, No Shear                 & 4.3242 & 4.3242 \\
$\theta \pm 0.4$, $T \pm 40$, \text{Shear} $= \pm 0.1$ & 5.5293 & 9.5700 \\
$\theta \pm 0.2$, $T \pm 20$, No Shear                 & 3.2817 & 3.2817 \\
$\theta \pm 0.2$, $T \pm 20$, \text{Shear} $= \pm 0.1$ & 4.5198 & 6.0682 \\\hline
\end{tabular}%
}
\caption{MRI-MRI registration results with affine model tuned on the datasets with the shown parameters.}
\label{table:mr_mr_affine_tuned_results}
\end{minipage}
\hfill
\begin{minipage}{0.48\textwidth}
\centering
\resizebox{\textwidth}{!}{%
\begin{tabular}{c|cc}\hline
Generation Parameters & $\|A\hat{B} - I\|$ & $\|B - \hat{B}\|$ \\\hline
$\theta \pm 0.4$, $T \pm 40$, No Shear                 & 3.6186 & 3.6186 \\
$\theta \pm 0.4$, $T \pm 40$, \text{Shear} $= \pm 0.1$ & 4.1670 & 9.1483 \\
$\theta \pm 0.2$, $T \pm 20$, No Shear                 & 2.7113 & 2.7113 \\
$\theta \pm 0.2$, $T \pm 20$, \text{Shear} $= \pm 0.1$ & 3.2596 & 5.2754 \\\hline
\end{tabular}%
}
\caption{CT-MRI registration results with affine model tuned on the datasets with the shown parameters.}
\label{table:ct_mr_affine_tuned_results}
\end{minipage}
\end{table}

\begin{table}[h!]
\centering
\begin{minipage}{0.48\textwidth}
\centering
\resizebox{\textwidth}{!}{%
\begin{tabular}{c|cc}\hline
Generation Parameters & $\|A\hat{B} - I\|$ & $\|B - \hat{B}\|$ \\\hline
$\theta \pm 0.4$, $T \pm 40$, No Shear                 & 3.4790  & 3.4790  \\
$\theta \pm 0.4$, $T \pm 40$, \text{Shear} $= \pm 0.1$ & 10.4729 & 12.3017 \\
$\theta \pm 0.2$, $T \pm 20$, No Shear                 & 3.0415  & 3.0415  \\
$\theta \pm 0.2$, $T \pm 20$, \text{Shear} $= \pm 0.1$ & 8.6077  & 9.3741  \\\hline
\end{tabular}%
}
\caption{MRI-MRI registration results with rigid model tuned on the datasets with the shown parameters.}
\label{table:mr_mr_rigid_tuned_results}
\end{minipage}
\hfill
\begin{minipage}{0.48\textwidth}
\centering
\resizebox{\textwidth}{!}{%
\begin{tabular}{c|cc}\hline
Generation Parameters & $\|A\hat{B} - I\|$ & $\|B - \hat{B}\|$ \\\hline
$\theta \pm 0.4$, $T \pm 40$, No Shear                 & 3.4099 & 3.4099 \\
$\theta \pm 0.4$, $T \pm 40$, \text{Shear} $= \pm 0.1$ & 5.8691 & 9.6244 \\
$\theta \pm 0.2$, $T \pm 20$, No Shear                 & 2.3015 & 2.3015 \\
$\theta \pm 0.2$, $T \pm 20$, \text{Shear} $= \pm 0.1$ & 4.4737 & 6.3426 \\\hline
\end{tabular}%
}
\caption{CT-MRI registration results with rigid model tuned on the datasets with the shown parameters.}
\label{table:ct_mr_rigid_tuned_results}
\end{minipage}
\end{table}
 % text inside


The results shown above are for each dataset, with the model trained on that dataset. While this can give a good overview of the performance of the model, it is not possible to make any statements about their relative performance without evaluating on the same things. In order to evaluate their overall performance we have chosen to evaluate them on just the two test datasets without shearing, with the aim of evaluating the performance on data that is as plausibly realistic as possible. Given that the $\|A\hat{B} - I\|$ and $\|B - \hat{B}\|$ metrics are so similar, we will include only the former. The results of these evaluations can be seen in Tables \ref{table:tuned_no_shear_AB_results} and \ref{table:tuned_no_shear_MI_results}. We also include the NMI metric here. In both these tables the ``Model Version'' describes whether the model in question is rigid or affine, and which training data was used to train it.

\begin{table}[h!]
\centering
\resizebox{\textwidth}{!}{%
\begin{tabular}{c|cccc|c}
\hline
$\|A\hat{B} - I\|$ & \multicolumn{4}{|c|}{Model Version} & Baseline \\ \hline
\multirow{3}{*}{Test Set parameters} & \multicolumn{4}{|c|}{Rigid} & Rigid \\ \cline{2-6}
      & \multicolumn{2}{c}{No Shear} & \multicolumn{2}{c|}{Shear$\pm0.1$} \\\cline{2-5}
      & $\theta\pm0.2$, $T\pm 20$ & $\theta\pm0.4$, $T\pm 40$ & $\theta\pm0.2$, $T\pm 20$ & $\theta\pm0.4$, $T\pm 40$ \\\cline{2-5}
      $\theta \pm 0.2$, $T \pm 20$, No Shear & 2.7420 & \textbf{2.6641} & 3.5810 & 3.9383 & 4.5827 \\
      $\theta \pm 0.4$, $T \pm 40$, No Shear & 6.8055 & 3.5053 & 8.3744 & 4.8060 & 8.8768\\ \cline{2-6}
      & \multicolumn{4}{|c|}{Affine} & Affine\\ \cline{2-6}
      $\theta \pm 0.2$, $T \pm 20$, No Shear & 2.7112 & 2.8222 & 2.9794 & 2.7970 & 3.8292\\
      $\theta \pm 0.4$, $T \pm 40$, No Shear & 5.6467 & \textbf{3.3947} & 6.7972 & 3.6661 & 7.4554\\\hline
\end{tabular}
}
\caption{CT-MRI registration results showing the $\|A\hat{B} - I\|$ loss of models trained on the different datasets. Rightmost column is MRI-MRI registration with baseline model for comparison. Bolded numbers denote the model with best performance for the relevant test set. Note that each test set here is spread out over two rows.}
\label{table:tuned_no_shear_AB_results}
\end{table}
%2413
%\end{table}

The NMI score shown here is, as mentioned earlier calculated using the MRI transformed alongside the moved CT scan that was used for the registration.

%%%%%%%%%%%%%%%%%%%%%%%%%%%%%%%%%%%%%%%%%%%%%%%%%%%%%%%%%%%%%%%%%%%%%%%%%%%%%%%%
\begin{table}[h!]
\centering
\resizebox{\textwidth}{!}{%
\begin{tabular}{c|cccc|c}
\hline
Normalised Mutual Information & \multicolumn{4}{|c|}{Model Version} & Baseline \\ \hline
\multirow{3}{*}{Test Set parameters} & \multicolumn{4}{|c|}{Rigid} & Rigid \\ \cline{2-6}
      & \multicolumn{2}{c}{No Shear} & \multicolumn{2}{c|}{Shear$\pm0.1$} \\\cline{2-5}
      & $\theta\pm0.2$, $T\pm 20$ & $\theta\pm0.4$, $T\pm 40$ & $\theta\pm0.2$, $T\pm 20$ & $\theta\pm0.4$, $T\pm 40$ \\\cline{2-5}
      $\theta \pm 0.2$, $T \pm 20$, No Shear & 0.2462 & 0.2464 & 0.2434 & 0.2426 & 0.2182 \\
      $\theta \pm 0.4$, $T \pm 40$, No Shear & 0.2430 & \textbf{0.2451} & 0.2349 & 0.2369 & 0.2137 \\\cline{2-6}
      & \multicolumn{4}{|c|}{Affine} & Affine\\ \cline{2-6}
      $\theta \pm 0.2$, $T \pm 20$, No Shear & 0.2475 & 0.2477 & 0.2476 & \textbf{0.2478} & 0.2055\\
      $\theta \pm 0.4$, $T \pm 40$, No Shear & 0.2441 & 0.2442 & 0.2444 & 0.2444 & 0.2017\\\hline
\end{tabular}%
}
\caption{CT-MRI registration results showing the MI score, for models trained on the different datasets. Rightmost column is MRI-MRI registration with baseline model for comparison. Bolded numbers denote the model with best performance for the relevant test set. Note that each test set here is spread out over two rows.}
\label{table:tuned_no_shear_MI_results}
\end{table}
%%%%%%%%%%%%%%%%%%%%%%%%%%%%%%%%%%%%%%%%%%%%%%%%%%%%%%%%%%%%%%%%%%%%%%%%%%%%%%%%

Examples showing the registration process can be seen in Figure \ref{fig:reg_example}. This is included to show that despite the NMI scores being in the range of roughly $0.20-0.22$, the resulting registration is visually very close, albeit with small differences. These differences can be visualised better in Figure \ref{fig:reg_diff}, which shows the absolute voxel errors of a single slice.

\begin{figure}[h!]
\centering
\includegraphics[width=0.90\textwidth]{images/registration_example_ct_mr.png}
\\
\includegraphics[width=0.90\textwidth]{images/registration_example_mr_mr.png}
\caption{Examples showing a 2D slice of a registered image. From left to right: moving, fixed, moved MRI, moved CT. The moving images have been subject to the exact same transform, for ease of comparison. The moved images in the top row have been moved according to the affine transform found by registering the moving CT to the fixed MRI. The bottom row is transformed according to the moving MRI.}\label{fig:reg_example}
\end{figure}

%Best model by dice: (4, (0.20270081025856937, 20.503577818261668))
% (0.4, 40, None, 'rigid') model
Our use of the SynthSeg for segmentation outputs a volume with 32 different classes. This is used as a tool to evaluate the clinical viability of the registration. An example showing a slice of segmented brains can be seen in Figure \ref{fig:seg_diff}.

\begin{figure}[h!]
  \centering
  \begin{minipage}{0.45\textwidth}
    \centering
    \includegraphics[width=0.48\textwidth]{images/registration_example_ct_mr_diff.png}
    \includegraphics[width=0.48\textwidth]{images/registration_example_mr_mr_diff.png}
    \caption{Example showing showing the same 2D slice as in Figure \ref{fig:reg_example}. Left image is CT-MRI registered, right is MRI-MRI registered. This shows the absolute differences between the fixed and moved MRI images.}\label{fig:reg_diff}
  \end{minipage}\hfill
  \begin{minipage}{0.45\textwidth}
    \centering
    \includegraphics[width=\textwidth]{images/segm_fixed_moved.png}
    \caption{Example showing a 2D slice of the segmentation. Left is ``ground truth'' segmentation of the fixed volume. Right is segmentation of moved image.}\label{fig:seg_diff}
  \end{minipage}
\end{figure}

\begin{figure}[h!]
    \centering
    \includegraphics[width=0.48\textwidth]{images/dice/data0.2_20_None_model0.4_40_0.1_rigid.png}
    \includegraphics[width=0.48\textwidth]{images/dice/data0.4_40_None_model0.4_40_None_rigid.png}
    \caption{Dice scores and Hausdorff distances of the SynthSeg segmentations of moved images using one of the best performing registration models.}\label{fig:dice_haus_example}
\end{figure}
