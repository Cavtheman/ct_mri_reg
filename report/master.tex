\documentclass{article}

% Custom .sty that handles a lot of nice things
% This includes importing a bunch of common packages
\usepackage{arxiv}

\title{
  CT-MRI cross-domain registration for better brain segmentation\\
  Master's Thesis
}

%\date{September 9, 1985}	% Here you can change the date presented in the paper title
%\date{} 					% Or removing it

\author{Casper Lisager Frandsen\\
  University of Copenhagen\\
  \texttt{clf@di.ku.dk}
  \AND           % You can comment this line and the next three if you're not in a group
  Mads Nielsen,\;\;Mostafa Mehdipour Ghazi\\
  University of Copenhagen, Pioneer Centre for AI\\
  \texttt{\{madsn,ghazi\}@di.ku.dk}
}

\renewcommand{\undertitle}{}

% Stylings for neat code in reports
% Feel free to add more language files (or replace these ones)
% These ones are all stolen from the internet somewhere
% python, fsharp, haskell
%\input{stylings/fsharp}
\input{stylings/python}
%\input{stylings/haskell}
% Write some code directly in latex
%\begin{lstlisting}
%  printf ("Hello World");
%\end{lstlisting}

% Show a full file. Path is relative from location of this master.tex file
% \lstinputlisting{path/to/my/file/my_file.fsx}

% Show lines 1-3
% \lstinputlisting[firstline=1, lastline=3]{path/to/my/file/my_file.fsx}

% Show 1-3 and 5-6 (also works with just one)
% Be mindful of editing code afterwards!
% \lstinputlisting[linerange=1-3, 5-6]{path/to/my/file/my_file.fsx}


\begin{document}
\maketitle
% Put all text below here

%\begin{abstract}
%	\lipsum[1]
%\end{abstract}
\tableofcontents
% Report (scientific paper):
%         Intro (1pg)
%         Methods
%            Data
%            Computational methods (often large part)
%            Eval criteria + statistics
%         Results
%            Data
%            Accuracy
%            Eval criteria + statistics
%         Discussion
%         (Limitations)
%         Conclusion
%
% MICCAI 2023 format is a potential style for report
% Write thesis such that it has:
%       Synopsis 4-10pg
%         Background
%         Motivation
%         Research question
%       Scientific paper
%       Supplementary material

% For ease of organisation, keep separate sections in separate files.
% This imports all latex code from latex/1.tex

\section{Introduction}
Computed Tomography (CT) and Magnetic Resonance Imaging (MRI) are widely utilised imaging techniques, used in medical diagnostics. CT scans utilise X-rays to create detailed images, and are particularly effective at visualising bones and internal injuries. These scans provide images of bony structures and are often used in cases including trauma due to the speed at which they can be taken.

MRI scans, on the other hand, use powerful superconducting electromagnets which can produce detailed images of soft tissues such as the brain, which we will be looking at in this paper. MRI is particularly effective for brain scans because it can differentiate between these soft tissues based on their varying water content and relaxation properties. In recent years, segmentation of MRI scans has become very accurate, and can be used to diagnose many health issues. However, MRI faces a few problems; namely that it is time consuming and expensive compared to other types of scan. As such, it is desirable for cheaper and quicker scans such as those made using CT to be automatically registered to an MRI, to leverage the strengths of both imaging methods. The purpose of the project is exactly that. If successful, this would allow such scans to take place in situations where time is of the essence and help medical personnel in taking more informed decisions.

There are many methods for registering images, with a commonly used one being the Elastix framework\cite{elastix}, which uses iterative methods. These methods are quite slow however, and deep learning-based methods such as Synthmorph have been proposed which drastically reduces registration time down to less than a minute on CPU and just a few seconds on GPU\cite{synthmorph}. Generally speaking, two different main types of registration exist and are currently used in the field, depending on the use cases, rigid and non-rigid. Rigid registration methods transform images based on constrained affine transformation matrix. The matrix is constrained such that it can only perform a subset of affine transformations, rotation and translation. This type of registration, despite being the ``weakest'' in terms of how much they can transform is nevertheless sufficient in many medical contexts.

Non-rigid registration includes both deformable and affine transformations. With deformable registration methods, each voxel can be individually transformed, effectively allowing transformations of inputs into completely unrecognisable outputs. Affine registration methods are essentially an extension of the rigid methods. While the affine transformation matrix used in rigid registration is constrained, it is when performing affine registration. This opens up the possibility to both scale, shear and reflect inputs, in addition to the aforementioned rigid capabilities.

Synthmorph consists of several different registration models, including both deformable, rigid and affine models. In this paper, we will be looking at only the rigid and affine versions of the models for registration of intra-subject pairs of CT and MRI scans.

All code used in this paper can be found on \href{https://github.com/Cavtheman/ct_mri_reg/}{Github}\footnote{\href{https://github.com/Cavtheman/ct_mri_reg/}{https://github.com/Cavtheman/ct\_mri\_reg/}}, though it has not as of the time of writing been made into a state usable for others.

\subsection{Motivation}

\section{Methods}
%\input{latex/1}
\subsection{Data}
For this project, the SynthRAD dataset\cite{synthradData} will serve as the primary resource. The SynthRAD dataset is a substantial collection of brain and pelvis CT and T1 MRI scans. Only the brain scans are used in this project. The data is also been completely anonymised, with all patient data removed.

The SynthRAD dataset is useful to this work, since it has been thoroughly preprocessed and registered using conventional (rigid) iterative methods. The methods by which this has been done can be read in the corresponding literature. This thorough preprocessing of the data is crucial, as it allows us in this paper to use the registered images as a ground truth for ``perfect'' registration later in this paper. As such, we will use the SynthRAD dataset for constructing the training, validation, and test sets used throughout this thesis.

It is noted in the original paper\cite{synthradData} that they have concentrated outliers from the norm in this training dataset, but have not provided their test set for public use, due to the fact that it was used in the SynthRAD2023 Grand Challenge. As such, these outliers may mean that the results shown in this paper are not entirely representative. However, this may have proven beneficial for training purposes, as the model should be less susceptible to these types of outliers.


\subsection{Artificial Data generation}
In the context of training a network for image registration, it is crucial to have access to relevant pairs of volumes that require registration. However, acquiring unregistered image pairs, particularly those with known and accurate registration transformations, presents significant challenges. To address this, we augment the existing SynthRAD dataset, which consists of 180 preprocessed and registered image pairs, by applying a limited set of transformations. While it may be safe to assume that the fixed scan that will be registered to is centred, we saw negligible difference in the baseline Synthmorph performance on non-centered data. Therefore, we have elected to transform both the fixed and moving volumes, providing better generalisation, and preventing the trained models from degenerating into moving the subject to the centre, regardless of input. For each pair of CT and T1 MRI scans, we output two different pairs of volumes, one pair marked and utilised as the fixed volumes, and one as the moving. For each CT-MRI pair, they are transformed identically, to keep them registered to each other. The fixed and moving pairs are transformed with identical parameters given to the random functions, but the fixed and moving transformations are chosen independently.

The limited set of transformations is used to preserve the clinical relevance of the original scans while introducing just enough variability to train the registration model effectively. This balance is vital to avoid creating synthetic data that might not be representative of actual medical images, which could lead to a model that performs well on our artificial data but fails in clinical practice, or learns irrelevant information.

Both the CT and MRI scans are normalised between 0 and 1 as part of preprocessing. While normalisation of data is common practice, it can be potentially problematic when working with CT scans. Unlike MRI scans, where intensity values are typically relative and vary based on tissue type and scan parameters, CT scans have intensity values corresponding to real physical phenomena. These values can be clinically meaningful and correspond to specific tissue densities, such as bone, air, and soft tissue. However, while the specific values are lost when normalised in this way, the relative values are not.

However, preliminary baseline experiments with the Synthmorph models showed that it performed markedly worse when normalisation was not performed. Given that this normalisation is performed in the original Synthmorph paper\cite{synthmorph}, we elected not to pursue this any further. As a result, all results not in the appendix are with normalised CT and T1 MRI scans, unless specifically noted otherwise. These baseline results can be seen in Tables \ref{appendix:mr_mr_results_affine}-\ref{appendix:ct_mr_results_rigid} in the appendix.

The remainder of the augmentation process involves applying standard rigid and affine transformations, which are varied across four generated datasets to evaluate the performance of different methods and models. The primary objective is to determine whether smaller, more clinically realistic transformations or larger, less realistic ones yield better results for training. Three distinct parameters are varied during training, each with separate components for each spatial dimension. However, not all aspects of the affine transformation space are utilised. Specifically, neither reflection nor scaling is applied.

Reflection is avoided due to the training regimen of the original Synthmorph model, which is trained to register volumes exclusively in the left-inferior-anterior (LIA) orientation. This model limitation was not well-documented, leading to significant challenges and time spent troubleshooting why the model initially failed when applied to the SynthRAD data, which is provided in a left-posterior-superior (LPS) orientation by default. Applying reflection would further exacerbate this issue, rendering the model non-functional. Additionally, given that the brain is not symmetrical in function, reflection could introduce additional inaccuracies when evaluating performance with the SynthSeg model\cite{synthseg1}\cite{synthseg2}.

Scaling is also omitted, as the Synthmorph model expects inputs to be standardised to 1mm$^3$ voxels. Any further scaling, in either direction, would likely degrade the model's performance without providing any practical benefit, as the scans can already be easily scaled to this voxel size. This decision ensures that the augmented data remains both realistic and compatible with the model's expectations.

Thus we are left with rotation, translation, and shear as the primary transformations employed in the augmentation process, though shear is not always applied. Each of these transformations includes components along all three spatial axes. Given that the scans are 3-dimensional, the affine matrices differ from the more commonly encountered 2-dimensional variants. Generalised 3D rotational matrices for each axis are shown below, each taking an angle $\theta$ in radians:

\begin{align*}
  R_x(\theta_x) =
  \begin{pmatrix}
    1 & 0 & 0 & 0 \\
    0 & \cos\theta_x & -\sin\theta_x & 0 \\
    0 & \sin\theta_x & \cos\theta_x & 0 \\
    0 & 0 & 0 & 1
  \end{pmatrix}
\end{align*}
\begin{align*}
  R_y(\theta_y) =
  \begin{pmatrix}
    \cos\theta_y & 0 & \sin\theta_y & 0 \\
    0 & 1 & 0 & 0 \\
    -\sin\theta_y & 0 & \cos\theta_y & 0 \\
    0 & 0 & 0 & 1
  \end{pmatrix}
\end{align*}
\begin{align*}
  R_z(\theta_z) =
  \begin{pmatrix}
    \cos\theta_z & -\sin\theta_z & 0 & 0 \\
    \sin\theta_z & \cos\theta_z & 0 & 0 \\
    0 & 0 & 1 & 0 \\
    0 & 0 & 0 & 1
  \end{pmatrix}
\end{align*}

For simplicity of implementation, these matrices are implemented separately. Since these are linear transformations, calculating the dot product of these matrices to achieve a singular rotation matrix is equivalent to performing each rotation separately. As such, we combine them, rotating around the $x$-axis first, then the $y$- and $z$-axes. For the data generation, values of $\theta$ are picked uniformly at random in the ranges $\pm0.2$ and $\pm0.4$ radians. Later in this paper, $\theta\pm0.2$ and $\theta\pm0.4$ will be used as shorthand to denote these distributions. $0.2$ radians corresponds roughly to $11$ and $22$ degrees respectively, chosen such that $\pm0.2$ represents a roughly realistic sample, where patient movement could plausibly be the cause of the extra rotation. The other, larger rotations are chosen to determine whether more variation will help the model generalise better. Since the point of this project is to demonstrate capabilities with CT scans, which may be used for more time-sensitive diagnoses operations, allowing more noise in the form of extra rotations should allow for more robust models. The resulting rotation matrix can be seen below. Just for ease of display here, the bottom row and rightmost column the matrix is removed, as it has no effect on rotation, and can be appended with no issue.

\begin{align}
  R &= R_z(\theta_z) R_y(\theta_y) R_x(\theta_x)\notag\\
  &=
  \begin{pmatrix}
    \cos\theta_y\cos\theta_z & \sin\theta_x\sin\theta_y\cos\theta_z \text{-} \cos\theta_x\sin\theta_z& \cos\theta_x\sin\theta_y\cos\theta_z \text{+} \sin\theta_x\sin\theta_z\\
    \cos\theta_y\sin\theta_z & \sin\theta_x\sin\theta_y\sin\theta_z \text{+} \cos\theta_x\cos\theta_z& \cos\theta_x\sin\theta_y\sin\theta_z \text{-} \sin\theta_x\cos\theta_z\\
    \text{-}\sin\theta_y & \sin\theta_x\cos\theta_z & \cos\theta_x\cos\theta_y
  \end{pmatrix}\label{eq:rot_matrix}
\end{align}

Simple translation is easier, simply corresponding to values in the rightmost column. Similar to rotation, values of $t$ are picked uniformly at random in the ranges $\pm20$ and $\pm40$, corresponding to $20$ and $40$ voxels respectively. Due to the 1mm$^3$ voxels, this corresponds to and equal amount of millimetres. $T\pm20$ and $T\pm40$ will be used as shorthand for this, like with rotation. Like with the rotational values, these values have been picked to represent a roughly realistic sample, and a more extreme example.

\begin{align*}
  T &=
  \begin{pmatrix}
    1 & 0 & 0 & t_x \\
    0 & 1 & 0 & t_y \\
    0 & 0 & 1 & t_z \\
    0 & 0 & 0 & 1
  \end{pmatrix}
\end{align*}

Affine shear matrices are similar to the rotational matrices in their makeup. Each non-diagonal value in the rotational part of the matrix corresponds to a shear over a specific axis.However, unlike rotational transformations, shear is not a phenomenon typically encountered in properly calibrated medical scanners. As a result, the values chosen for shear in this study are intentionally smaller. And indeed, for half the datasets generated in this paper, no shear is applied. For the other half, shear is applied in the same fashion as rotation and translation, chosen uniformly at random in the interval $\pm0.1$. The reasoning behind applying shear to this data, is that we are training both a rigid and affine transformation model to register, and the extra parameters the affine model has to work with may allow for it to more easily converge on useful solutions, even if it does not need to make non-rigid solutions. Nonetheless, both models, rigid and affine, are trained on datasets with and without shear.

\begin{align*}
  S &=
  \begin{pmatrix}
    1 & s_{xy} & s_{xz} & 0 \\
    s_{yx} & 1 & s_{yz} & 0 \\
    s_{zx} & s_{zy} & 1 & 0 \\
    0 & 0 & 0 & 1
  \end{pmatrix}
\end{align*}

One problem arises when applying these transformations however, and that is the fact that the rotation is around coordinate $(0,0,0)$. In the ideal case, that would be the middle of the scan, but that is not the case here, where $(0,0,0)$ is a corner of the volume. To correct for this fact, changes need to be made to the translational matrix, to effectively change the centre of rotation. Note that $R$ in this refers specifically to the form given in Equation \ref{eq:rot_matrix}, without a fourth column or row. $|V_{x,y,z}|$ refers to the size of the scan volume, in each axis, in voxels.

\begin{align*}
  \begin{pmatrix}
    t_x'\\
    t_y'\\
    t_z'
  \end{pmatrix}
  &=
  \begin{pmatrix}
    t_x\\
    t_y\\
    t_z
  \end{pmatrix}
  -
  \left(
  \begin{pmatrix}
    0.5|V_x| \\
    0.5|V_y| \\
    0.5|V_z| \\
  \end{pmatrix}
  -
  R
  \begin{pmatrix}
    0.5|V_x| \\
    0.5|V_y| \\
    0.5|V_z| \\
  \end{pmatrix}
  \right)\\
  T' &=
  \begin{pmatrix}
    1 & 0 & 0 & t_x' \\
    0 & 1 & 0 & t_y' \\
    0 & 0 & 1 & t_z' \\
    0 & 0 & 0 & 1
  \end{pmatrix}
\end{align*}

The final affine matrix $A$ used to transform the volumes is then calculated by taking the dot product of all these, where $R_{full}$ denotes $R$ with an extra column and row, to be a valid affine matrix:
\begin{align*}
  A = S\;R_{full}\;T'
\end{align*}

\begin{table}[h!]
\centering
\begin{tabular}{c|ccc}
\hline
Dataset Parameters                          & Training Set & Validation Set & Test Set \\\hline
$\theta \pm 0.4$, $T \pm 40$, No Shear      & 720          & 180            & 180      \\
$\theta \pm 0.4$, $T \pm 40$, Shear$\pm0.1$ & 720          & 180            & 180      \\
$\theta \pm 0.2$, $T \pm 20$, No Shear      & 720          & 180            & 180      \\
$\theta \pm 0.2$, $T \pm 20$, Shear$\pm0.1$ & 720          & 180            & 180      \\\hline
\end{tabular}
\caption{Size of various Training, Validation, and Test Sets}
\label{table:data_sizes}
\end{table}

With all of these transformations, we have generated a training set containing 720 pairs of scans, while validation and test sets each consist of just 180 pairs for each set of generation parameters. With this distribution of data, each training set contains four versions of each original image, each transformed differently, while the test and validation data contain exactly one transformed original.



%A matrix like this can be inverted analytically, and this will be used to create a metric for determining the closeness of the predicted transformations by the Synthmorph model. The inversion requires a bit of explanation however. Consider the general affine transformation matrix:
%\begin{align*}
%  A =
%  \begin{pmatrix}
%    \begin{array}{c|c}
%      R & t \\
%      \hline
%      0 & 1 \\
%    \end{array}
%  \end{pmatrix}
%  &=
%  \begin{pmatrix}
%    a_{11} & a_{12} & a_{13} & t_x \\
%    a_{21} & a_{22} & a_{23} & t_y \\
%    a_{31} & a_{32} & a_{33} & t_z \\
%    0 & 0 & 0 & 1
%  \end{pmatrix}
%\end{align*}
%
%Where $R$ corresponds to the rotation matrix and $t$ corresponds to the vector $(t_x,t_y,t_z)^T$. Because $R$ is unitary, its transpose is equivalent to its inverse. The inverse of a translation is simply the negative, however the inverse rotation must also be applied. Therefore, the inverse of the original affine matrix is structured as follows:
%
%\begin{align*}
%  A^{-1} =
%  \begin{pmatrix}
%    \begin{array}{c|c}
%      R^T & R^T(-t) \\
%      \hline
%      0 & 1 \\
%    \end{array}
%  \end{pmatrix}
%\end{align*}


\subsection{Methods}
\subsection{Evaluation Criteria}\label{method:criteria}
Synthseg segmentation\cite{synthseg1}\cite{synthseg2}
\subsection{Statistics}



\section{Results}
To demonstrate the improvements made here over the capabilities of original Synthmorph model, we have first evaluated its performance on all four test datasets without any fine-tuning. However, for purposes of making the results more medically plausible, most of the models have not been evaluated on the test sets containing images that have been sheared, and metrics have not been calculated for them on the test sets. It will be clearly denoted where the shear test sets have been used.

% Baseline synthmorph
As is to be expected, Synthmorph without any tuning performs very poorly when trying to register CT to MRI scans. This can be seen in Tables \ref{table:aff_ct_mr_baseline} and \ref{table:rigid_ct_mr_baseline} where both the $\|A\hat{B} - I\|$ and $\|B - \hat{B}\|$ loss values are roughly 2-5 times larger than the corresponding loss values of MRI-MRI registration in Tables \ref{table:aff_mr_mr_baseline} and \ref{table:rigid_mr_mr_baseline}.

%%%%%%%%%%%%%%%%%%%%%%%%%%%%%%%%%%%%%%%%%%%%%%%%%%%%%%%%%%%%%%%%%%%%%%%%%%%%%%%%
% Updated affine SynthMorph results
\begin{table}[h!]
\centering
\begin{minipage}{0.48\textwidth}
\centering
\resizebox{\textwidth}{!}{%
\begin{tabular}{c|cc}\hline
Generation Parameters & $\|A\hat{B} - I\|$ & $\|B - \hat{B}\|$ \\\hline
$\theta \pm 0.4$, $T \pm 40$, No Shear                 & 7.4554  & 7.4554  \\
$\theta \pm 0.4$, $T \pm 40$, \text{Shear} $= \pm 0.1$ & 7.5588  & 7.5306  \\
$\theta \pm 0.2$, $T \pm 20$, No Shear                 & 3.8292  & 3.8292  \\
$\theta \pm 0.2$, $T \pm 20$, \text{Shear} $= \pm 0.1$ & 4.4211  & 4.4242  \\\hline
\end{tabular}%
}
\caption{MRI-MRI registration results with different transformations using the untuned affine SynthMorph model.}
\label{table:aff_mr_mr_baseline}
\end{minipage}
\hfill
\begin{minipage}{0.48\textwidth}
\centering
\resizebox{\textwidth}{!}{%
\begin{tabular}{c|cc}\hline
Generation Parameters & $\|A\hat{B} - I\|$ & $\|B - \hat{B}\|$ \\\hline
$\theta \pm 0.4$, $T \pm 40$, No Shear                 & 44.1560 & 44.1560 \\
$\theta \pm 0.4$, $T \pm 40$, \text{Shear} $= \pm 0.1$ & 44.5351 & 44.1592 \\
$\theta \pm 0.2$, $T \pm 20$, No Shear                 & 42.0576 & 42.0576 \\
$\theta \pm 0.2$, $T \pm 20$, \text{Shear} $= \pm 0.1$ & 43.3569 & 42.9668 \\\hline
\end{tabular}%
}
\caption{CT-MRI registration results with different transformations using the untuned affine SynthMorph model.}
\label{table:aff_ct_mr_baseline}
\end{minipage}
\end{table}

% Updated rigid SynthMorph results
\begin{table}[h!]
\centering
\begin{minipage}{0.48\textwidth}
\centering
\resizebox{\textwidth}{!}{%
\begin{tabular}{c|cc}\hline
Generation Parameters & $\|A\hat{B} - I\|$ & $\|B - \hat{B}\|$ \\\hline
$\theta \pm 0.4$, $T \pm 40$, No Shear                 & 8.8768  & 8.8768  \\
$\theta \pm 0.4$, $T \pm 40$, \text{Shear} $= \pm 0.1$ & 19.1687 & 19.1985 \\
$\theta \pm 0.2$, $T \pm 20$, No Shear                 & 4.5827  & 4.5827  \\
$\theta \pm 0.2$, $T \pm 20$, \text{Shear} $= \pm 0.1$ & 16.8906 & 16.8533 \\\hline
\end{tabular}%
}
\caption{MRI-MRI registration results with different transformations using the untuned rigid SynthMorph model.}
\label{table:rigid_mr_mr_baseline}
\end{minipage}
\hfill
\begin{minipage}{0.48\textwidth}
\centering
\resizebox{\textwidth}{!}{%
\begin{tabular}{c|cc}\hline
Generation Parameters & $\|A\hat{B} - I\|$ & $\|B - \hat{B}\|$ \\\hline
$\theta \pm 0.4$, $T \pm 40$, No Shear                 & 24.9288 & 24.9288 \\
$\theta \pm 0.4$, $T \pm 40$, \text{Shear} $= \pm 0.1$ & 31.9698 & 32.7524 \\
$\theta \pm 0.2$, $T \pm 20$, No Shear                 & 23.8434 & 23.8434 \\
$\theta \pm 0.2$, $T \pm 20$, \text{Shear} $= \pm 0.1$ & 29.9778 & 30.7856 \\\hline
\end{tabular}%
}
\caption{CT-MRI registration results with different transformations using the untuned rigid SynthMorph model.}
\label{table:rigid_ct_mr_baseline}
\end{minipage}
\end{table}
 % text inside
After training both the affine and rigids versions of the Synthmorph model on our augmented datasets, we find that the network starts outperforming the baseline model. For MRI-MRI registration this can easily be explained via slight overfitting to the dataset at hand. This can be seen in Tables \ref{table:mr_mr_affine_tuned_results} and \ref{table:mr_mr_rigid_tuned_results}.

For the CT-MRI results we also see significant improvements thanks to the training. The $\|A\hat{B} - I\|$ and $\|B - \hat{B}\|$ loss values can be seen in Tables \ref{table:ct_mr_affine_tuned_results} and \ref{table:ct_mr_rigid_tuned_results} to have improved to a point where the CT-MRI registrations perform roughly equivalently to the MRI-MRI registration.

% Tuned with $\|A\hat{B} - I\|$ loss
\begin{table}[h!]
\centering
\begin{minipage}{0.48\textwidth}
\centering
\resizebox{\textwidth}{!}{%
\begin{tabular}{c|cc}\hline
Generation Parameters & $\|A\hat{B} - I\|$ & $\|B - \hat{B}\|$ \\\hline
$\theta \pm 0.4$, $T \pm 40$, No Shear                 & 4.3242 & 4.3242 \\
$\theta \pm 0.4$, $T \pm 40$, \text{Shear} $= \pm 0.1$ & 5.5293 & 9.5700 \\
$\theta \pm 0.2$, $T \pm 20$, No Shear                 & 3.2817 & 3.2817 \\
$\theta \pm 0.2$, $T \pm 20$, \text{Shear} $= \pm 0.1$ & 4.5198 & 6.0682 \\\hline
\end{tabular}%
}
\caption{MRI-MRI registration results with affine model tuned on the datasets with the shown parameters.}
\label{table:mr_mr_affine_tuned_results}
\end{minipage}
\hfill
\begin{minipage}{0.48\textwidth}
\centering
\resizebox{\textwidth}{!}{%
\begin{tabular}{c|cc}\hline
Generation Parameters & $\|A\hat{B} - I\|$ & $\|B - \hat{B}\|$ \\\hline
$\theta \pm 0.4$, $T \pm 40$, No Shear                 & 3.6186 & 3.6186 \\
$\theta \pm 0.4$, $T \pm 40$, \text{Shear} $= \pm 0.1$ & 4.1670 & 9.1483 \\
$\theta \pm 0.2$, $T \pm 20$, No Shear                 & 2.7113 & 2.7113 \\
$\theta \pm 0.2$, $T \pm 20$, \text{Shear} $= \pm 0.1$ & 3.2596 & 5.2754 \\\hline
\end{tabular}%
}
\caption{CT-MRI registration results with affine model tuned on the datasets with the shown parameters.}
\label{table:ct_mr_affine_tuned_results}
\end{minipage}
\end{table}

\begin{table}[h!]
\centering
\begin{minipage}{0.48\textwidth}
\centering
\resizebox{\textwidth}{!}{%
\begin{tabular}{c|cc}\hline
Generation Parameters & $\|A\hat{B} - I\|$ & $\|B - \hat{B}\|$ \\\hline
$\theta \pm 0.4$, $T \pm 40$, No Shear                 & 3.4790  & 3.4790  \\
$\theta \pm 0.4$, $T \pm 40$, \text{Shear} $= \pm 0.1$ & 10.4729 & 12.3017 \\
$\theta \pm 0.2$, $T \pm 20$, No Shear                 & 3.0415  & 3.0415  \\
$\theta \pm 0.2$, $T \pm 20$, \text{Shear} $= \pm 0.1$ & 8.6077  & 9.3741  \\\hline
\end{tabular}%
}
\caption{MRI-MRI registration results with rigid model tuned on the datasets with the shown parameters.}
\label{table:mr_mr_rigid_tuned_results}
\end{minipage}
\hfill
\begin{minipage}{0.48\textwidth}
\centering
\resizebox{\textwidth}{!}{%
\begin{tabular}{c|cc}\hline
Generation Parameters & $\|A\hat{B} - I\|$ & $\|B - \hat{B}\|$ \\\hline
$\theta \pm 0.4$, $T \pm 40$, No Shear                 & 3.4099 & 3.4099 \\
$\theta \pm 0.4$, $T \pm 40$, \text{Shear} $= \pm 0.1$ & 5.8691 & 9.6244 \\
$\theta \pm 0.2$, $T \pm 20$, No Shear                 & 2.3015 & 2.3015 \\
$\theta \pm 0.2$, $T \pm 20$, \text{Shear} $= \pm 0.1$ & 4.4737 & 6.3426 \\\hline
\end{tabular}%
}
\caption{CT-MRI registration results with rigid model tuned on the datasets with the shown parameters.}
\label{table:ct_mr_rigid_tuned_results}
\end{minipage}
\end{table}
 % text inside


The results shown above are for each dataset, with the model trained on that dataset. While this can give a good overview of the performance of the model, it is not possible to make any statements about their relative performance without evaluating on the same things. In order to evaluate their overall performance we have chosen to evaluate them on just the two test datasets without shearing, with the aim of evaluating the performance on data that is as plausibly realistic as possible. Given that the $\|A\hat{B} - I\|$ and $\|B - \hat{B}\|$ metrics are so similar, we will include only the former. The results of these evaluations can be seen in Tables \ref{table:tuned_no_shear_AB_results} and \ref{table:tuned_no_shear_MI_results}. We also include the NMI metric here. In both these tables the ``Model Version'' describes whether the model in question is rigid or affine, and which training data was used to train it.

\begin{table}[h!]
\centering
\resizebox{\textwidth}{!}{%
\begin{tabular}{c|cccc|c}
\hline
$\|A\hat{B} - I\|$ & \multicolumn{4}{|c|}{Model Version} & Baseline \\ \hline
\multirow{3}{*}{Test Set parameters} & \multicolumn{4}{|c|}{Rigid} & Rigid \\ \cline{2-6}
      & \multicolumn{2}{c}{No Shear} & \multicolumn{2}{c|}{Shear$\pm0.1$} \\\cline{2-5}
      & $\theta\pm0.2$, $T\pm 20$ & $\theta\pm0.4$, $T\pm 40$ & $\theta\pm0.2$, $T\pm 20$ & $\theta\pm0.4$, $T\pm 40$ \\\cline{2-5}
      $\theta \pm 0.2$, $T \pm 20$, No Shear & 2.7420 & \textbf{2.6641} & 3.5810 & 3.9383 & 4.5827 \\
      $\theta \pm 0.4$, $T \pm 40$, No Shear & 6.8055 & 3.5053 & 8.3744 & 4.8060 & 8.8768\\ \cline{2-6}
      & \multicolumn{4}{|c|}{Affine} & Affine\\ \cline{2-6}
      $\theta \pm 0.2$, $T \pm 20$, No Shear & 2.7112 & 2.8222 & 2.9794 & 2.7970 & 3.8292\\
      $\theta \pm 0.4$, $T \pm 40$, No Shear & 5.6467 & \textbf{3.3947} & 6.7972 & 3.6661 & 7.4554\\\hline
\end{tabular}
}
\caption{CT-MRI registration results showing the $\|A\hat{B} - I\|$ loss of models trained on the different datasets. Rightmost column is MRI-MRI registration with baseline model for comparison. Bolded numbers denote the model with best performance for the relevant test set. Note that each test set here is spread out over two rows.}
\label{table:tuned_no_shear_AB_results}
\end{table}
%2413
%\end{table}

The NMI score shown here is, as mentioned earlier calculated using the MRI transformed alongside the moved CT scan that was used for the registration.

%%%%%%%%%%%%%%%%%%%%%%%%%%%%%%%%%%%%%%%%%%%%%%%%%%%%%%%%%%%%%%%%%%%%%%%%%%%%%%%%
\begin{table}[h!]
\centering
\resizebox{\textwidth}{!}{%
\begin{tabular}{c|cccc|c}
\hline
Normalised Mutual Information & \multicolumn{4}{|c|}{Model Version} & Baseline \\ \hline
\multirow{3}{*}{Test Set parameters} & \multicolumn{4}{|c|}{Rigid} & Rigid \\ \cline{2-6}
      & \multicolumn{2}{c}{No Shear} & \multicolumn{2}{c|}{Shear$\pm0.1$} \\\cline{2-5}
      & $\theta\pm0.2$, $T\pm 20$ & $\theta\pm0.4$, $T\pm 40$ & $\theta\pm0.2$, $T\pm 20$ & $\theta\pm0.4$, $T\pm 40$ \\\cline{2-5}
      $\theta \pm 0.2$, $T \pm 20$, No Shear & 0.2462 & 0.2464 & 0.2434 & 0.2426 & 0.2182 \\
      $\theta \pm 0.4$, $T \pm 40$, No Shear & 0.2430 & \textbf{0.2451} & 0.2349 & 0.2369 & 0.2137 \\\cline{2-6}
      & \multicolumn{4}{|c|}{Affine} & Affine\\ \cline{2-6}
      $\theta \pm 0.2$, $T \pm 20$, No Shear & 0.2475 & 0.2477 & 0.2476 & \textbf{0.2478} & 0.2055\\
      $\theta \pm 0.4$, $T \pm 40$, No Shear & 0.2441 & 0.2442 & 0.2444 & 0.2444 & 0.2017\\\hline
\end{tabular}%
}
\caption{CT-MRI registration results showing the MI score, for models trained on the different datasets. Rightmost column is MRI-MRI registration with baseline model for comparison. Bolded numbers denote the model with best performance for the relevant test set. Note that each test set here is spread out over two rows.}
\label{table:tuned_no_shear_MI_results}
\end{table}
%%%%%%%%%%%%%%%%%%%%%%%%%%%%%%%%%%%%%%%%%%%%%%%%%%%%%%%%%%%%%%%%%%%%%%%%%%%%%%%%

Examples showing the registration process can be seen in Figure \ref{fig:reg_example}. This is included to show that despite the NMI scores being in the range of roughly $0.20-0.22$, the resulting registration is visually very close, albeit with small differences. These differences can be visualised better in Figure \ref{fig:reg_diff}, which shows the absolute voxel errors of a single slice.

\begin{figure}[h!]
\centering
\includegraphics[width=0.90\textwidth]{images/registration_example_ct_mr.png}
\\
\includegraphics[width=0.90\textwidth]{images/registration_example_mr_mr.png}
\caption{Examples showing a 2D slice of a registered image. From left to right: moving, fixed, moved MRI, moved CT. The moving images have been subject to the exact same transform, for ease of comparison. The moved images in the top row have been moved according to the affine transform found by registering the moving CT to the fixed MRI. The bottom row is transformed according to the moving MRI.}\label{fig:reg_example}
\end{figure}

%Best model by dice: (4, (0.20270081025856937, 20.503577818261668))
% (0.4, 40, None, 'rigid') model
Our use of the SynthSeg for segmentation outputs a volume with 32 different classes. This is used as a tool to evaluate the clinical viability of the registration. An example showing a slice of segmented brains can be seen in Figure \ref{fig:seg_diff}.

\begin{figure}[h!]
  \centering
  \begin{minipage}{0.45\textwidth}
    \centering
    \includegraphics[width=0.48\textwidth]{images/registration_example_ct_mr_diff.png}
    \includegraphics[width=0.48\textwidth]{images/registration_example_mr_mr_diff.png}
    \caption{Example showing showing the same 2D slice as in Figure \ref{fig:reg_example}. Left image is CT-MRI registered, right is MRI-MRI registered. This shows the absolute differences between the fixed and moved MRI images.}\label{fig:reg_diff}
  \end{minipage}\hfill
  \begin{minipage}{0.45\textwidth}
    \centering
    \includegraphics[width=\textwidth]{images/segm_fixed_moved.png}
    \caption{Example showing a 2D slice of the segmentation. Left is ``ground truth'' segmentation of the fixed volume. Right is segmentation of moved image.}\label{fig:seg_diff}
  \end{minipage}
\end{figure}

\begin{figure}[h!]
    \centering
    \includegraphics[width=0.48\textwidth]{images/dice/data0.2_20_None_model0.4_40_0.1_rigid.png}
    \includegraphics[width=0.48\textwidth]{images/dice/data0.4_40_None_model0.4_40_None_rigid.png}
    \caption{Dice scores and Hausdorff distances of the SynthSeg segmentations of moved images using one of the best performing registration models.}\label{fig:dice_haus_example}
\end{figure}

\subsection{Data} % Probably put these in results .tex file
%\input{latex/1}
\subsection{Accuracy}
%\input{latex/1}
\subsection{Evaluation Criteria}
%\input{latex/1}
\subsection{Statistics}
%\input{latex/1}



\section{Discussion}
%\input{latex/1}
\subsection{Limitations}



\section{Conclusion}
%\input{latex/1}

%footnote example: \footnote{Text here will appear in footnote}


% Neat bibliography
% Feel free to uncomment if you refer to books, articles etc.
\bibliographystyle{unsrtnat}
\bibliography{references}

\end{document}
