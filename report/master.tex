\documentclass{article}

% Custom .sty that handles a lot of nice things
% This includes importing a bunch of common packages
\usepackage{arxiv}

\title{
  CT-MRI cross-domain registration for better brain segmentation%\\
  %Master's Thesis
}

%\date{September 9, 1985}	% Here you can change the date presented in the paper title
%\date{} 					% Or removing it

\author{Casper Lisager Frandsen\\
  University of Copenhagen\\
  \texttt{clf@di.ku.dk}
  \AND           % You can comment this line and the next three if you're not in a group
  Mads Nielsen,\;\;Mostafa Mehdipour Ghazi\\
  University of Copenhagen, Pioneer Centre for AI\\
  \texttt{\{madsn,ghazi\}@di.ku.dk}
}

\renewcommand{\undertitle}{}

% Stylings for neat code in reports
% Feel free to add more language files (or replace these ones)
% These ones are all stolen from the internet somewhere
% python, fsharp, haskell
%\usepackage{color}
\usepackage{listings}
\usepackage{upquote}
\definecolor{bluekeywords}{rgb}{0.13,0.13,1}
\definecolor{greencomments}{rgb}{0,0.5,0}
\definecolor{redstrings}{rgb}{0.9,0,0}


\lstdefinelanguage{FSharp}%
{morekeywords={let, new, match, with, rec, open, module, namespace, type, of, member, %
and, for, while, true, false, in, do, begin, end, fun, function, return, yield, try, %
mutable, if, then, else, cloud, async, static, use, abstract, interface, inherit, finally },
  otherkeywords={ let!, return!, do!, yield!, use!, var, from, select, where, order, by },
  keywordstyle=\color{bluekeywords},
  sensitive=true,
  basicstyle=\ttfamily,
	breaklines=true,
  xleftmargin=\parindent,
  aboveskip=\bigskipamount,
	tabsize=4,
  morecomment=[l][\color{greencomments}]{///},
  morecomment=[l][\color{greencomments}]{//},
  morecomment=[s][\color{greencomments}]{{(*}{*)}},
  morestring=[b]",
  showstringspaces=false,
  literate={`}{\`}1,
  stringstyle=\color{redstrings},
}
\lstset{
  language=FSharp
}

%Brugseksempler:

% Skriv koden direkte i LaTeX
% \begin{lstlisting}
%   printfn "Hello World"
% \end{lstlisting}

% Viser en hel fil. Stien starter fra folderen som .tex filen ligger i
% \lstinputlisting{path/to/my/file/my_file.fsx}

% Viser linjerne 1-3
% \lstinputlisting[firstline=1, lastline=3]{path/to/my/file/my_file.fsx}

% Viser linjerne 1-3 og 5-6
% \lstinputlisting[linerange=1-3, 5-6]{path/to/my/file/my_file.fsx}

\usepackage{xcolor}  % For a colorfull presentation
\usepackage{listings}  % For presenting code

% Definition of a style for code, matter of taste
\lstdefinestyle{mystyle}{
  language=Python,
  basicstyle=\ttfamily\footnotesize,
  backgroundcolor=\color[HTML]{F7F7F7},
  rulecolor=\color[HTML]{EEEEEE},
  identifierstyle=\color[HTML]{24292E},
  emphstyle=\color[HTML]{005CC5},
  keywordstyle=\color[HTML]{D73A49},
  commentstyle=\color[HTML]{6A737D},
  stringstyle=\color[HTML]{032F62},
  emph={@property,self,range,True,False},
  morekeywords={super,with,as,lambda},
  literate=%
    {+}{{{\color[HTML]{D73A49}+}}}1
    {-}{{{\color[HTML]{D73A49}-}}}1
    {*}{{{\color[HTML]{D73A49}*}}}1
    {/}{{{\color[HTML]{D73A49}/}}}1
    {=}{{{\color[HTML]{D73A49}=}}}1
    {/=}{{{\color[HTML]{D73A49}=}}}1,
  breakatwhitespace=false,
  breaklines=true,
  captionpos=b,
  keepspaces=true,
  numbers=none,
  showspaces=false,
  showstringspaces=false,
  showtabs=false,
  tabsize=4,
  frame=single,
}
\lstset{style=mystyle}

%\usepackage{color}
\usepackage{listings}
\usepackage{upquote}
\definecolor{bluekeywords}{rgb}{0.13,0.13,1}
\definecolor{greencomments}{rgb}{0,0.5,0}
\definecolor{redstrings}{rgb}{0.9,0,0}


\lstset{
  frame=none,
  xleftmargin=2pt,
  stepnumber=1,
  numbers=left,
  numbersep=5pt,
  numberstyle=\ttfamily\tiny\color[gray]{0.3},
  belowcaptionskip=\bigskipamount,
  captionpos=b,
  escapeinside={*'}{'*},
  language=haskell,
  tabsize=2,
  emphstyle={\bf},
  commentstyle=\it\color{greencomments},
  stringstyle=\mdseries\rmfamily\color{redstrings},
  showspaces=false,
  keywordstyle=\bfseries\rmfamily\color{bluekeywords},
  columns=flexible,
  basicstyle=\small\sffamily,
  showstringspaces=false,
  morecomment=[l]\%,
}

% Write some code directly in latex
%\begin{lstlisting}
%  printf ("Hello World");
%\end{lstlisting}

% Show a full file. Path is relative from location of this master.tex file
% \lstinputlisting{path/to/my/file/my_file.fsx}

% Show lines 1-3
% \lstinputlisting[firstline=1, lastline=3]{path/to/my/file/my_file.fsx}

% Show 1-3 and 5-6 (also works with just one)
% Be mindful of editing code afterwards!
% \lstinputlisting[linerange=1-3, 5-6]{path/to/my/file/my_file.fsx}


\begin{document}
\maketitle

% Put all text below here
\begin{abstract} % unfinished
  This paper investigates an extension of the SynthMorph network (Hoffmann et al. 2022)\cite{synthmorphModel} with additional capabilities, namely cross-domain registration. The network is originally designed for multi-modal MRI registration, but in this paper we train it to also handle CT to MRI registration. We train the network to do this using only the affine transformation matrices generated by the network. This CT-MRI registration capability could help combine the rapid imaging capabilities of CT scans, and the soft tissue details of MRI scans.

  While our findings suggest that the SynthMorph shows promise for cross-modal registration, we believe it will likely need further refinement to be a clinically viable tool. We show that currently the performance does not live up to the standards required in practice yet.
  %This paper investigates the extension of the SynthMorph network, originally designed for multi-modal MRI registration, to handle cross-domain CT-MRI registration. We evaluate the performance of SynthMorph in registering CT scans to T1 MRI scans, aiming to leverage the rapid imaging capabilities of CT and the detailed soft tissue contrast of MRI. Despite the successful application of SynthMorph to various MRI modalities, our results indicate that the network, without fine-tuning, performs significantly worse in CT-MRI registration. Specifically, our fine-tuned model achieved a mean Dice score of 0.3312 in white matter segmentation, compared to the over 0.6 Dice score reported in the original SynthMorph paper for MRI-MRI registrations. This discrepancy highlights the challenges inherent in cross-domain registration, particularly in clinical applications. While our tuned models demonstrated improvements, the overall performance suggests that further refinement and perhaps novel approaches may be required to meet clinical standards. Future work should explore the generalization of this model to other imaging modalities and investigate techniques to mitigate model-induced hallucinations, particularly in the context of diffusion models.
\end{abstract}
\newpage
\tableofcontents
% Report (scientific paper):
%         Intro (1pg)
%         Methods
%            Data
%            Computational methods (often large part)
%            Eval criteria + statistics
%         Results
%            Data
%            Accuracy
%            Eval criteria + statistics
%         Discussion
%         (Limitations)
%         Conclusion
%
% MICCAI 2023 format is a potential style for report
% Write thesis such that it has:
%       Synopsis 4-10pg
%         Background
%         Motivation
%         Research question
%       Scientific paper
%       Supplementary material

% For ease of organisation, keep separate sections in separate files.
% This imports all latex code from latex/1.tex

% Nicole:
% 1. Introduction
% 2. Related work
% 3. Proposed Method
% 3.1 study data
% 3.2 seg network
% 3.3 reg network
% 3.4 metrics
% 4. Experiments and results
% 4.1 seg
% 4.2 reg
% 5. Discussion
% 6. Conclusion

\section{Introduction}
In recent years, the segmentation of magnetic resonance imaging (MRI) has become very accurate, and can be used to diagnose many health issues. However, MRI faces a few problems; namely that it is time consuming and expensive compared to other scans. As such, it is desirable for cheaper and quicker scans such as computed tomography (CT) to be automatically registered to an MRI.\@ The purpose of the project is exactly that. If successful, this would allow such scans to take place in situations where time is of the essence and help medical personnel in taking more informed decisions.
 % finished

\section{Methods}
\subsection{Data}
For this project, I will primarily be using the SynthRAD\cite{synthradData} dataset, which is a relatively large collection of 3D CT and T1 MRI scan pairs. That is, this dataset consists of 180 pairs of CT and MRI scans which are already preprocessed and registered, making it a potentially incredibly useful dataset for training models on.
 % finished
\subsection{Models}
\subsubsection{Registration Network}
The basis of the model fine-tuned in this paper is the SynthMorph model developed by Hoffmann et al. (2023)\cite{synthmorphModel}. There are several variants of SynthMorph, including one that generates a dense displacement field. In this paper however, we focus on the variants that produce rigid and affine transformations. Since we are concerned here with pairs of scans from the same patient, any non-rigid deformation is very likely to be incorrect.

The original Synthmorph is trained using purely synthetic images, generated from anatomical label maps. The synthetic images are varied significantly during training to cover a wide range of real-world scenarios, including different image contrasts and resolutions. This approach enables the base network to effectively handle various modalities, such as T1 and T2 MRI scans. The goal of our work is to extend this capability beyond just multi-modal compatibility to achieve robust cross-domain performance as well.

The actual network architecture is fully convolutional, utilising eight 3D convolutional layers, which output spatial feature maps separately for the moving and fixed images. The barycentres of these feature maps are then calculated, and the points are used to analytically align the two images with weighted-least-squares.

%Training on Synthetic Data: The model is trained using a novel approach where synthetic images are generated from anatomical label maps. These synthetic images are intentionally varied to cover a wide range of potential real-world scenarios, including different imaging contrasts and resolutions. This allows the model to learn registration in a way that is robust to changes in acquisition parameters, such as MRI contrast or noise.

%Anatomy-Aware Registration: Unlike traditional methods that may require preprocessing steps like skull stripping to focus on the relevant anatomy, SynthMorph incorporates anatomy-specific learning directly into the model. By optimizing the overlap of selected anatomical labels (such as white matter, gray matter, and cerebrospinal fluid), the model becomes adept at aligning the relevant structures while ignoring irrelevant image content.

%Affine Transform Prediction: The model predicts an affine transform that aligns the input image pair by fitting the transformation to match the anatomical features. This involves computing barycenters and weights for each predicted feature map, and then aligning these using a weighted least-squares approach.

%Implementation Details: The model architecture uses a convolutional neural network (CNN) to extract features from the images, with the output being a full affine transformation.


\subsubsection{Segmentation Network}
The SynthSeg network, by Billot et al. (2023)\cite{synthseg1}\cite{synthseg2} is also used in this paper. This network is used out of the box however, and is not fine-tuned. Instead, it is utilised to provide a less abstract, and more clinically relevant and interpretable metric for final evaluation of the fine-tuned registration models. The manner in which this is done is detailed in the Evaluation Criteria subsection

The SynthSeg model is a 3D UNet\cite{unet} type network, trained on synthetic data like the Synthmorph network. The synthetic data used to train this network is generated based on a corresponding synthetic label map, that is itself generated by a transformation of a label map from the training data. This approach allows the model to generalise well over different image types, including T1 scans as utilised in this paper.
 % unfinished?
\subsection{Artificial Data Generation}
In the context of training a network for image registration, it is crucial to have access to relevant pairs of volumes that require registration. However, acquiring unregistered image pairs, particularly those with known and accurate registration transformations, presents significant challenges. To address this, we augment the existing SynthRAD dataset, which consists of 180 preprocessed and registered image pairs, by applying a limited set of transformations. While it may be safe to assume that the fixed scan that will be registered to is centred, we saw negligible difference in the baseline Synthmorph performance on non-centered data. Therefore, we have elected to transform both the fixed and moving volumes, providing better generalisation, and preventing the trained models from degenerating into moving the subject to the centre, regardless of input. For each pair of CT and T1 MRI scans, we output two different pairs of volumes, one pair marked and utilised as the fixed volumes, and one as the moving. For each CT-MRI pair, they are transformed identically, to keep them registered to each other. The fixed and moving pairs are transformed with identical parameters given to the random functions, but the fixed and moving transformations are chosen independently.

The limited set of transformations is used to preserve the clinical relevance of the original scans while introducing just enough variability to train the registration model effectively. This balance is vital to avoid creating synthetic data that might not be representative of actual medical images, which could lead to a model that performs well on our artificial data but fails in clinical practice, or learns irrelevant information.

Both the CT and MRI scans are normalised between 0 and 1 as part of preprocessing. While normalisation of data is common practice, it can be potentially problematic when working with CT scans. Unlike MRI scans, where intensity values are typically relative and vary based on tissue type and scan parameters, CT scans have intensity values corresponding to real physical phenomena. These values can be clinically meaningful and correspond to specific tissue densities, such as bone, air, and soft tissue. However, while the specific values are lost when normalised in this way, the relative values are not.

However, preliminary baseline experiments with the Synthmorph models showed that it performed markedly worse when normalisation was not performed. Given that this normalisation is performed in the original Synthmorph paper\cite{synthmorph}, we elected not to pursue this any further. As a result, all results not in the appendix are with normalised CT and T1 MRI scans, unless specifically noted otherwise. These baseline results can be seen in Tables \ref{appendix:mr_mr_results_affine}-\ref{appendix:ct_mr_results_rigid} in the appendix.

The remainder of the augmentation process involves applying standard rigid and affine transformations, which are varied across four generated datasets to evaluate the performance of different methods and models. The primary objective is to determine whether smaller, more clinically realistic transformations or larger, less realistic ones yield better results for training. Three distinct parameters are varied during training, each with separate components for each spatial dimension. However, not all aspects of the affine transformation space are utilised. Specifically, neither reflection nor scaling is applied.

Reflection is avoided due to the training regimen of the original Synthmorph model, which is trained to register volumes exclusively in the left-inferior-anterior (LIA) orientation. This model limitation was not well-documented, leading to significant challenges and time spent troubleshooting why the model initially failed when applied to the SynthRAD data, which is provided in a left-posterior-superior (LPS) orientation by default. Applying reflection would further exacerbate this issue, rendering the model non-functional. Additionally, given that the brain is not symmetrical in function, reflection could introduce additional inaccuracies when evaluating performance with the SynthSeg model\cite{synthseg1}\cite{synthseg2}.

Scaling is also omitted, as the Synthmorph model expects inputs to be standardised to 1mm$^3$ voxels. Any further scaling, in either direction, would likely degrade the model's performance without providing any practical benefit, as the scans can already be easily scaled to this voxel size. This decision ensures that the augmented data remains both realistic and compatible with the model's expectations.

Thus we are left with rotation, translation, and shear as the primary transformations employed in the augmentation process, though shear is not always applied. Each of these transformations includes components along all three spatial axes. Given that the scans are 3-dimensional, the affine matrices differ from the more commonly encountered 2-dimensional variants. Generalised 3D rotational matrices for each axis are shown below, each taking an angle $\theta$ in radians:

\begin{align*}
  R_x(\theta_x) =
  \begin{pmatrix}
    1 & 0 & 0 & 0 \\
    0 & \cos\theta_x & -\sin\theta_x & 0 \\
    0 & \sin\theta_x & \cos\theta_x & 0 \\
    0 & 0 & 0 & 1
  \end{pmatrix}
\end{align*}
\begin{align*}
  R_y(\theta_y) =
  \begin{pmatrix}
    \cos\theta_y & 0 & \sin\theta_y & 0 \\
    0 & 1 & 0 & 0 \\
    -\sin\theta_y & 0 & \cos\theta_y & 0 \\
    0 & 0 & 0 & 1
  \end{pmatrix}
\end{align*}
\begin{align*}
  R_z(\theta_z) =
  \begin{pmatrix}
    \cos\theta_z & -\sin\theta_z & 0 & 0 \\
    \sin\theta_z & \cos\theta_z & 0 & 0 \\
    0 & 0 & 1 & 0 \\
    0 & 0 & 0 & 1
  \end{pmatrix}
\end{align*}

For simplicity of implementation, these matrices are implemented separately. Since these are linear transformations, calculating the dot product of these matrices to achieve a singular rotation matrix is equivalent to performing each rotation separately. As such, we combine them, rotating around the $x$-axis first, then the $y$- and $z$-axes. For the data generation, values of $\theta$ are picked uniformly at random in the ranges $\pm0.2$ and $\pm0.4$ radians. Later in this paper, $\theta\pm0.2$ and $\theta\pm0.4$ will be used as shorthand to denote these distributions. $0.2$ radians corresponds roughly to $11$ and $22$ degrees respectively, chosen such that $\pm0.2$ represents a roughly realistic sample, where patient movement could plausibly be the cause of the extra rotation. The other, larger rotations are chosen to determine whether more variation will help the model generalise better. Since the point of this project is to demonstrate capabilities with CT scans, which may be used for more time-sensitive diagnoses operations, allowing more noise in the form of extra rotations should allow for more robust models. The resulting rotation matrix can be seen below. Just for ease of display here, the bottom row and rightmost column the matrix is removed, as it has no effect on rotation, and can be appended with no issue.

\begin{align}
  R &= R_z(\theta_z) R_y(\theta_y) R_x(\theta_x)\notag\\
  &=
  \begin{pmatrix}
    \cos\theta_y\cos\theta_z & \sin\theta_x\sin\theta_y\cos\theta_z \text{-} \cos\theta_x\sin\theta_z& \cos\theta_x\sin\theta_y\cos\theta_z \text{+} \sin\theta_x\sin\theta_z\\
    \cos\theta_y\sin\theta_z & \sin\theta_x\sin\theta_y\sin\theta_z \text{+} \cos\theta_x\cos\theta_z& \cos\theta_x\sin\theta_y\sin\theta_z \text{-} \sin\theta_x\cos\theta_z\\
    \text{-}\sin\theta_y & \sin\theta_x\cos\theta_z & \cos\theta_x\cos\theta_y
  \end{pmatrix}\label{eq:rot_matrix}
\end{align}

Simple translation is easier, simply corresponding to values in the rightmost column. Similar to rotation, values of $t$ are picked uniformly at random in the ranges $\pm20$ and $\pm40$, corresponding to $20$ and $40$ voxels respectively. Due to the 1mm$^3$ voxels, this corresponds to and equal amount of millimetres. $T\pm20$ and $T\pm40$ will be used as shorthand for this, like with rotation. Like with the rotational values, these values have been picked to represent a roughly realistic sample, and a more extreme example.

\begin{align*}
  T &=
  \begin{pmatrix}
    1 & 0 & 0 & t_x \\
    0 & 1 & 0 & t_y \\
    0 & 0 & 1 & t_z \\
    0 & 0 & 0 & 1
  \end{pmatrix}
\end{align*}

Affine shear matrices are similar to the rotational matrices in their makeup. Each non-diagonal value in the rotational part of the matrix corresponds to a shear over a specific axis.However, unlike rotational transformations, shear is not a phenomenon typically encountered in properly calibrated medical scanners. As a result, the values chosen for shear in this study are intentionally smaller. And indeed, for half the datasets generated in this paper, no shear is applied. For the other half, shear is applied in the same fashion as rotation and translation, chosen uniformly at random in the interval $\pm0.1$. The reasoning behind applying shear to this data, is that we are training both a rigid and affine transformation model to register, and the extra parameters the affine model has to work with may allow for it to more easily converge on useful solutions, even if it does not need to make non-rigid solutions. Nonetheless, both models, rigid and affine, are trained on datasets with and without shear.

\begin{align*}
  S &=
  \begin{pmatrix}
    1 & s_{xy} & s_{xz} & 0 \\
    s_{yx} & 1 & s_{yz} & 0 \\
    s_{zx} & s_{zy} & 1 & 0 \\
    0 & 0 & 0 & 1
  \end{pmatrix}
\end{align*}

One problem arises when applying these transformations however, and that is the fact that the rotation is around coordinate $(0,0,0)$. In the ideal case, that would be the middle of the scan, but that is not the case here, where $(0,0,0)$ is a corner of the volume. To correct for this fact, changes need to be made to the translational matrix, to effectively change the centre of rotation. Note that $R$ in this refers specifically to the form given in Equation \ref{eq:rot_matrix}, without a fourth column or row. $|V_{x,y,z}|$ refers to the size of the scan volume, in each axis, in voxels.

\begin{align*}
  \begin{pmatrix}
    t_x'\\
    t_y'\\
    t_z'
  \end{pmatrix}
  &=
  \begin{pmatrix}
    t_x\\
    t_y\\
    t_z
  \end{pmatrix}
  -
  \left(
  \begin{pmatrix}
    0.5|V_x| \\
    0.5|V_y| \\
    0.5|V_z| \\
  \end{pmatrix}
  -
  R
  \begin{pmatrix}
    0.5|V_x| \\
    0.5|V_y| \\
    0.5|V_z| \\
  \end{pmatrix}
  \right)\\
  T' &=
  \begin{pmatrix}
    1 & 0 & 0 & t_x' \\
    0 & 1 & 0 & t_y' \\
    0 & 0 & 1 & t_z' \\
    0 & 0 & 0 & 1
  \end{pmatrix}
\end{align*}

The final affine matrix $A$ used to transform the volumes is then calculated by taking the dot product of all these, where $R_{full}$ denotes $R$ with an extra column and row, to be a valid affine matrix:
\begin{align*}
  A = S\;R_{full}\;T'
\end{align*}

\begin{table}[h!]
\centering
\begin{tabular}{c|ccc}
\hline
Dataset Parameters                          & Training Set & Validation Set & Test Set \\\hline
$\theta \pm 0.4$, $T \pm 40$, No Shear      & 720          & 180            & 180      \\
$\theta \pm 0.4$, $T \pm 40$, Shear$\pm0.1$ & 720          & 180            & 180      \\
$\theta \pm 0.2$, $T \pm 20$, No Shear      & 720          & 180            & 180      \\
$\theta \pm 0.2$, $T \pm 20$, Shear$\pm0.1$ & 720          & 180            & 180      \\\hline
\end{tabular}
\caption{Size of various Training, Validation, and Test Sets}
\label{table:data_sizes}
\end{table}

With all of these transformations, we have generated a training set containing 720 pairs of scans, while validation and test sets each consist of just 180 pairs for each set of generation parameters. With this distribution of data, each training set contains four versions of each original image, each transformed differently, while the test and validation data contain exactly one transformed original.



%A matrix like this can be inverted analytically, and this will be used to create a metric for determining the closeness of the predicted transformations by the Synthmorph model. The inversion requires a bit of explanation however. Consider the general affine transformation matrix:
%\begin{align*}
%  A =
%  \begin{pmatrix}
%    \begin{array}{c|c}
%      R & t \\
%      \hline
%      0 & 1 \\
%    \end{array}
%  \end{pmatrix}
%  &=
%  \begin{pmatrix}
%    a_{11} & a_{12} & a_{13} & t_x \\
%    a_{21} & a_{22} & a_{23} & t_y \\
%    a_{31} & a_{32} & a_{33} & t_z \\
%    0 & 0 & 0 & 1
%  \end{pmatrix}
%\end{align*}
%
%Where $R$ corresponds to the rotation matrix and $t$ corresponds to the vector $(t_x,t_y,t_z)^T$. Because $R$ is unitary, its transpose is equivalent to its inverse. The inverse of a translation is simply the negative, however the inverse rotation must also be applied. Therefore, the inverse of the original affine matrix is structured as follows:
%
%\begin{align*}
%  A^{-1} =
%  \begin{pmatrix}
%    \begin{array}{c|c}
%      R^T & R^T(-t) \\
%      \hline
%      0 & 1 \\
%    \end{array}
%  \end{pmatrix}
%\end{align*}
 % finished
\subsection{Evaluation Criteria}
The training process for the Synthmorph model in this paper is not performed in the same was as it was originally trained. Instead of working with completely synthetic images and computing the Dice loss over corresponding label maps, the training performed in this paper uses real, but affinely transformed images as data. This lets us ignore any label maps during training, and use a much simpler loss function. When generating transformed images, both the affine matrix ($A$) and its inverse ($B$) used to generate it are saved for later use. Say that a volume in the training set is transformed from its original using an affine matrix $A$, and its inverse matrix $B$ is generated alongside it. Then the following statements will hold:
\begin{align*}
  AB &= I\\
  AB - I &=
  \begin{pmatrix}
    0&0&0&0\\
    0&0&0&0\\
    0&0&0&0\\
    0&0&0&0
  \end{pmatrix}\\
  \|AB - I\| &= 0
\end{align*}
With this setup, given the fixed volume, and the volume transformed by affine matrix $A$, Synthmorph will then output an affine matrix $\hat{B}$. This means we can then use the $\|A\hat{B} - I\|$ as a measure of how close $\hat{B}$ is to the inverse of $A$. Importantly, this also means that no output volume needs to be generated, greatly speeding up training time. This approach simplifies training, by focusing on the affine transformations rather than relying on labeled data, while still ensuring that the transformations can be accurately reversed.

An additional similar metric is also used in this paper to evaluate performance, but is not used during training. It is based off of similar principles, leading to the loss function $\|B-\hat{B}\|$.
\begin{align*}
  B-B &=
  \begin{pmatrix}
    0&0&0&0\\
    0&0&0&0\\
    0&0&0&0\\
    0&0&0&0
  \end{pmatrix}\\
  \|B-B\| &= 0
\end{align*}

Both of these metrics are quite simple, and make for easy fine-tuning of the Synthmorph models. However, while they model differences and errors in transformations well, they do not account for anatomical details. Therefore, we have also included a variant of the Mutual Information (MI) metric where it is normalised between $0$ and $1$, which has been demonstrated to be a good metric for registration\cite{mi}, taking the actual transformed outputs of Synthmorph and comparing to each other.

Mutual Information is a measure based in the field of information theory, which can quantify the amount of information able to be obtained from one random variable through another. In an image registration context, this is done by evaluating how much knowing the intensity distribution of one image reduces uncertainty about the intensity distribution of another. When working with continuous intensity values, these are put in several intensity bins and treated as equivalent. For this, the entropy function is important:
\begin{align*}
  H(A) &= -\sum_ip(A_i)\text{log}p(A_i)
\end{align*}
Where $p(A_i)$ denotes the probability of a pixel having intensity $A_i$, summing over all intensity bins. With these bins, the MI is then calculated using the following formula:
\begin{align*}
  \text{MI}(A,B) &= H(A) + H(B) - H(A,B)
\end{align*}
However, this measure can grow potentially infinitely, so we want to normalise it for easier training. This is done as follows:
\begin{align*}
  \text{NMI}(A,B) &= \frac{2\cdot\text{MI}(A,B)}{H(A)+H(B)}
\end{align*}
This Normalised Mutual Information (NMI) is thus one of the metrics that will be used in the rest of this paper to evaluate the performance of the registration. For the purposes of this paper, the NMI is always calculated between the moved and fixed MRI scans. This is also true when evaluating CT-MRI registration, as there is always an MRI counterpart to the transformed CT scan that has been transformed using the same parameters.

The Dice metric is utilised in this paper in much the same manner as the original Synthmorph paper, but is only used for evaluation purposes. However, since the SynthRAD data has not been segmented, we use the SynthSeg model to provide a segmentation of the fixed image, which is then used as the ground truth. When evaluating the registration performance, SynthSeg is also used to segment the moved image. These two segmentations are then used to calculate the Dice score, providing a measure of how well the registration has preserved anatomical structures.

It should be noted that there are a lot of classes in the SynthSeg output, and as such the Dice score is calculated for each one separately. Doing it this way means that we can read any voxels with the given class a True value, and any other as False, repeating over all classes. Given two volumes, A and B, we can then calculate the Dice score as follows:
\begin{align}
\text{Dice}(A, B) &= \frac{2 \times |A \cap B|}{|A| + |B|}
\end{align}\label{eq:dice}

%If there is complete overlap between the two segmentations, then we should expect a Dice score of 1.
 % finished?

\subsection{Training Process}
\begin{figure}
\centering
\captionsetup{justification=centering}
\includegraphics[width=0.23\textwidth]{images/loss_0.2_20_0.1_rigid.png}
\includegraphics[width=0.23\textwidth]{images/loss_0.2_20_None_rigid.png}
\includegraphics[width=0.23\textwidth]{images/loss_0.2_20_0.1_affine.png}
\includegraphics[width=0.23\textwidth]{images/loss_0.2_20_None_affine.png}
\\
\includegraphics[width=0.23\textwidth]{images/loss_0.4_40_0.1_rigid.png}
\includegraphics[width=0.23\textwidth]{images/loss_0.4_40_None_rigid.png}
\includegraphics[width=0.23\textwidth]{images/loss_0.4_40_0.1_affine.png}
\includegraphics[width=0.23\textwidth]{images/loss_0.4_40_None_affine.png}
\caption{Training loss for fine-tuning both the rigid and affine Synthmorph models, on datasets with various generation parameters.}\label{fig:training_loss}
\end{figure}

Training of the of the Synthmorph network in this paper is very much standard, by the book. After the augmented datasets have been generated, the CT and MRI pairs are fed to the model, using the specified $\|A\hat{B} - I\|$ loss, using a learning rate of 1e-4. For each of the datasets, two models are trained for comparison; one rigid and one affine. All of the models are trained on the training dataset detailed in Table \ref{table:data_sizes} for 20 epochs, and validated between each epoch. A checkpoint model is also saved for each epoch, and after training is finished, the model that performed best on the validation dataset is used for further testing and final evaluation. The training process of each of these models can be seen in Figure \ref{fig:training_loss}. The models that performed best on the validation set can be seen in Table \ref{table:chosen_models}

\begin{table}[h!]
  \centering
  \begin{tabular}{c|cc}

    \multirow{2}{*}{Dataset Parameters}         & \multicolumn{2}{|c}{Epoch} \\\cline{2-3}
                                                & Rigid model & Affine model \\\hline
    $\theta \pm 0.4$, $T \pm 40$, No Shear      & 19          & 18           \\
    $\theta \pm 0.4$, $T \pm 40$, Shear$\pm0.1$ & 19          & 19           \\
    $\theta \pm 0.2$, $T \pm 20$, No Shear      & 18          & 18           \\
    $\theta \pm 0.2$, $T \pm 20$, Shear$\pm0.1$ & 19          & 19           \\\hline
  \end{tabular}
  \caption{Best performing models for each dataset}
  \label{table:chosen_models}
\end{table}

%"aug_data/norm_rot0.4_trans40_shearNone/finetune_rigid/synthmorph_epoch19.h5"
%"aug_data/norm_rot0.4_trans40_shear0.1/finetune_rigid/synthmorph_epoch19.h5"
%"aug_data/norm_rot0.2_trans20_shearNone/finetune_rigid/synthmorph_epoch18.h5"
%"aug_data/norm_rot0.2_trans20_shear0.1/finetune_rigid/synthmorph_epoch19.h5"

%"aug_data/norm_rot0.4_trans40_shearNone/finetune_affine/synthmorph_epoch18.h5"
%"aug_data/norm_rot0.4_trans40_shear0.1/finetune_affine/synthmorph_epoch19.h5"
%"aug_data/norm_rot0.2_trans20_shearNone/finetune_affine/synthmorph_epoch18.h5"
%"aug_data/norm_rot0.2_trans20_shear0.1/finetune_affine/synthmorph_epoch19.h5"
 % finished?
%\subsection{Evaluation Criteria}\label{method:criteria}
%Synthseg segmentation\cite{synthseg1}\cite{synthseg2}

%\subsection{Statistics}


\section{Experiments and Results}
To demonstrate the improvements made here over the capabilities of original Synthmorph model, we have first evaluated its performance on all four test datasets without any fine-tuning. However, for purposes of making the results more medically plausible, most of the models have not been evaluated on the test sets containing images that have been sheared, and metrics have not been calculated for them on the test sets. It will be clearly denoted where the shear test sets have been used.

% Baseline synthmorph
As is to be expected, Synthmorph without any tuning performs very poorly when trying to register CT to MRI scans. This can be seen in Tables \ref{table:aff_ct_mr_baseline} and \ref{table:rigid_ct_mr_baseline} where both the $\|A\hat{B} - I\|$ and $\|B - \hat{B}\|$ loss values are roughly 2-5 times larger than the corresponding loss values of MRI-MRI registration in Tables \ref{table:aff_mr_mr_baseline} and \ref{table:rigid_mr_mr_baseline}.

%%%%%%%%%%%%%%%%%%%%%%%%%%%%%%%%%%%%%%%%%%%%%%%%%%%%%%%%%%%%%%%%%%%%%%%%%%%%%%%%
% Updated affine SynthMorph results
\begin{table}[h!]
\centering
\begin{minipage}{0.48\textwidth}
\centering
\resizebox{\textwidth}{!}{%
\begin{tabular}{c|cc}\hline
Generation Parameters & $\|A\hat{B} - I\|$ & $\|B - \hat{B}\|$ \\\hline
$\theta \pm 0.4$, $T \pm 40$, No Shear                 & 7.4554  & 7.4554  \\
$\theta \pm 0.4$, $T \pm 40$, \text{Shear} $= \pm 0.1$ & 7.5588  & 7.5306  \\
$\theta \pm 0.2$, $T \pm 20$, No Shear                 & 3.8292  & 3.8292  \\
$\theta \pm 0.2$, $T \pm 20$, \text{Shear} $= \pm 0.1$ & 4.4211  & 4.4242  \\\hline
\end{tabular}%
}
\caption{MRI-MRI registration results with different transformations using the untuned affine SynthMorph model.}
\label{table:aff_mr_mr_baseline}
\end{minipage}
\hfill
\begin{minipage}{0.48\textwidth}
\centering
\resizebox{\textwidth}{!}{%
\begin{tabular}{c|cc}\hline
Generation Parameters & $\|A\hat{B} - I\|$ & $\|B - \hat{B}\|$ \\\hline
$\theta \pm 0.4$, $T \pm 40$, No Shear                 & 44.1560 & 44.1560 \\
$\theta \pm 0.4$, $T \pm 40$, \text{Shear} $= \pm 0.1$ & 44.5351 & 44.1592 \\
$\theta \pm 0.2$, $T \pm 20$, No Shear                 & 42.0576 & 42.0576 \\
$\theta \pm 0.2$, $T \pm 20$, \text{Shear} $= \pm 0.1$ & 43.3569 & 42.9668 \\\hline
\end{tabular}%
}
\caption{CT-MRI registration results with different transformations using the untuned affine SynthMorph model.}
\label{table:aff_ct_mr_baseline}
\end{minipage}
\end{table}

% Updated rigid SynthMorph results
\begin{table}[h!]
\centering
\begin{minipage}{0.48\textwidth}
\centering
\resizebox{\textwidth}{!}{%
\begin{tabular}{c|cc}\hline
Generation Parameters & $\|A\hat{B} - I\|$ & $\|B - \hat{B}\|$ \\\hline
$\theta \pm 0.4$, $T \pm 40$, No Shear                 & 8.8768  & 8.8768  \\
$\theta \pm 0.4$, $T \pm 40$, \text{Shear} $= \pm 0.1$ & 19.1687 & 19.1985 \\
$\theta \pm 0.2$, $T \pm 20$, No Shear                 & 4.5827  & 4.5827  \\
$\theta \pm 0.2$, $T \pm 20$, \text{Shear} $= \pm 0.1$ & 16.8906 & 16.8533 \\\hline
\end{tabular}%
}
\caption{MRI-MRI registration results with different transformations using the untuned rigid SynthMorph model.}
\label{table:rigid_mr_mr_baseline}
\end{minipage}
\hfill
\begin{minipage}{0.48\textwidth}
\centering
\resizebox{\textwidth}{!}{%
\begin{tabular}{c|cc}\hline
Generation Parameters & $\|A\hat{B} - I\|$ & $\|B - \hat{B}\|$ \\\hline
$\theta \pm 0.4$, $T \pm 40$, No Shear                 & 24.9288 & 24.9288 \\
$\theta \pm 0.4$, $T \pm 40$, \text{Shear} $= \pm 0.1$ & 31.9698 & 32.7524 \\
$\theta \pm 0.2$, $T \pm 20$, No Shear                 & 23.8434 & 23.8434 \\
$\theta \pm 0.2$, $T \pm 20$, \text{Shear} $= \pm 0.1$ & 29.9778 & 30.7856 \\\hline
\end{tabular}%
}
\caption{CT-MRI registration results with different transformations using the untuned rigid SynthMorph model.}
\label{table:rigid_ct_mr_baseline}
\end{minipage}
\end{table}
 % text inside
After training both the affine and rigids versions of the Synthmorph model on our augmented datasets, we find that the network starts outperforming the baseline model. For MRI-MRI registration this can easily be explained via slight overfitting to the dataset at hand. This can be seen in Tables \ref{table:mr_mr_affine_tuned_results} and \ref{table:mr_mr_rigid_tuned_results}.

For the CT-MRI results we also see significant improvements thanks to the training. The $\|A\hat{B} - I\|$ and $\|B - \hat{B}\|$ loss values can be seen in Tables \ref{table:ct_mr_affine_tuned_results} and \ref{table:ct_mr_rigid_tuned_results} to have improved to a point where the CT-MRI registrations perform roughly equivalently to the MRI-MRI registration.

% Tuned with $\|A\hat{B} - I\|$ loss
\begin{table}[h!]
\centering
\begin{minipage}{0.48\textwidth}
\centering
\resizebox{\textwidth}{!}{%
\begin{tabular}{c|cc}\hline
Generation Parameters & $\|A\hat{B} - I\|$ & $\|B - \hat{B}\|$ \\\hline
$\theta \pm 0.4$, $T \pm 40$, No Shear                 & 4.3242 & 4.3242 \\
$\theta \pm 0.4$, $T \pm 40$, \text{Shear} $= \pm 0.1$ & 5.5293 & 9.5700 \\
$\theta \pm 0.2$, $T \pm 20$, No Shear                 & 3.2817 & 3.2817 \\
$\theta \pm 0.2$, $T \pm 20$, \text{Shear} $= \pm 0.1$ & 4.5198 & 6.0682 \\\hline
\end{tabular}%
}
\caption{MRI-MRI registration results with affine model tuned on the datasets with the shown parameters.}
\label{table:mr_mr_affine_tuned_results}
\end{minipage}
\hfill
\begin{minipage}{0.48\textwidth}
\centering
\resizebox{\textwidth}{!}{%
\begin{tabular}{c|cc}\hline
Generation Parameters & $\|A\hat{B} - I\|$ & $\|B - \hat{B}\|$ \\\hline
$\theta \pm 0.4$, $T \pm 40$, No Shear                 & 3.6186 & 3.6186 \\
$\theta \pm 0.4$, $T \pm 40$, \text{Shear} $= \pm 0.1$ & 4.1670 & 9.1483 \\
$\theta \pm 0.2$, $T \pm 20$, No Shear                 & 2.7113 & 2.7113 \\
$\theta \pm 0.2$, $T \pm 20$, \text{Shear} $= \pm 0.1$ & 3.2596 & 5.2754 \\\hline
\end{tabular}%
}
\caption{CT-MRI registration results with affine model tuned on the datasets with the shown parameters.}
\label{table:ct_mr_affine_tuned_results}
\end{minipage}
\end{table}

\begin{table}[h!]
\centering
\begin{minipage}{0.48\textwidth}
\centering
\resizebox{\textwidth}{!}{%
\begin{tabular}{c|cc}\hline
Generation Parameters & $\|A\hat{B} - I\|$ & $\|B - \hat{B}\|$ \\\hline
$\theta \pm 0.4$, $T \pm 40$, No Shear                 & 3.4790  & 3.4790  \\
$\theta \pm 0.4$, $T \pm 40$, \text{Shear} $= \pm 0.1$ & 10.4729 & 12.3017 \\
$\theta \pm 0.2$, $T \pm 20$, No Shear                 & 3.0415  & 3.0415  \\
$\theta \pm 0.2$, $T \pm 20$, \text{Shear} $= \pm 0.1$ & 8.6077  & 9.3741  \\\hline
\end{tabular}%
}
\caption{MRI-MRI registration results with rigid model tuned on the datasets with the shown parameters.}
\label{table:mr_mr_rigid_tuned_results}
\end{minipage}
\hfill
\begin{minipage}{0.48\textwidth}
\centering
\resizebox{\textwidth}{!}{%
\begin{tabular}{c|cc}\hline
Generation Parameters & $\|A\hat{B} - I\|$ & $\|B - \hat{B}\|$ \\\hline
$\theta \pm 0.4$, $T \pm 40$, No Shear                 & 3.4099 & 3.4099 \\
$\theta \pm 0.4$, $T \pm 40$, \text{Shear} $= \pm 0.1$ & 5.8691 & 9.6244 \\
$\theta \pm 0.2$, $T \pm 20$, No Shear                 & 2.3015 & 2.3015 \\
$\theta \pm 0.2$, $T \pm 20$, \text{Shear} $= \pm 0.1$ & 4.4737 & 6.3426 \\\hline
\end{tabular}%
}
\caption{CT-MRI registration results with rigid model tuned on the datasets with the shown parameters.}
\label{table:ct_mr_rigid_tuned_results}
\end{minipage}
\end{table}
 % text inside


The results shown above are for each dataset, with the model trained on that dataset. While this can give a good overview of the performance of the model, it is not possible to make any statements about their relative performance without evaluating on the same things. In order to evaluate their overall performance we have chosen to evaluate them on just the two test datasets without shearing, with the aim of evaluating the performance on data that is as plausibly realistic as possible. Given that the $\|A\hat{B} - I\|$ and $\|B - \hat{B}\|$ metrics are so similar, we will include only the former. The results of these evaluations can be seen in Tables \ref{table:tuned_no_shear_AB_results} and \ref{table:tuned_no_shear_MI_results}. We also include the NMI metric here. In both these tables the ``Model Version'' describes whether the model in question is rigid or affine, and which training data was used to train it.

\begin{table}[h!]
\centering
\resizebox{\textwidth}{!}{%
\begin{tabular}{c|cccc|c}
\hline
$\|A\hat{B} - I\|$ & \multicolumn{4}{|c|}{Model Version} & Baseline \\ \hline
\multirow{3}{*}{Test Set parameters} & \multicolumn{4}{|c|}{Rigid} & Rigid \\ \cline{2-6}
      & \multicolumn{2}{c}{No Shear} & \multicolumn{2}{c|}{Shear$\pm0.1$} \\\cline{2-5}
      & $\theta\pm0.2$, $T\pm 20$ & $\theta\pm0.4$, $T\pm 40$ & $\theta\pm0.2$, $T\pm 20$ & $\theta\pm0.4$, $T\pm 40$ \\\cline{2-5}
      $\theta \pm 0.2$, $T \pm 20$, No Shear & 2.7420 & \textbf{2.6641} & 3.5810 & 3.9383 & 4.5827 \\
      $\theta \pm 0.4$, $T \pm 40$, No Shear & 6.8055 & 3.5053 & 8.3744 & 4.8060 & 8.8768\\ \cline{2-6}
      & \multicolumn{4}{|c|}{Affine} & Affine\\ \cline{2-6}
      $\theta \pm 0.2$, $T \pm 20$, No Shear & 2.7112 & 2.8222 & 2.9794 & 2.7970 & 3.8292\\
      $\theta \pm 0.4$, $T \pm 40$, No Shear & 5.6467 & \textbf{3.3947} & 6.7972 & 3.6661 & 7.4554\\\hline
\end{tabular}
}
\caption{CT-MRI registration results showing the $\|A\hat{B} - I\|$ loss of models trained on the different datasets. Rightmost column is MRI-MRI registration with baseline model for comparison. Bolded numbers denote the model with best performance for the relevant test set. Note that each test set here is spread out over two rows.}
\label{table:tuned_no_shear_AB_results}
\end{table}
%2413
%\end{table}

The NMI score shown here is, as mentioned earlier calculated using the MRI transformed alongside the moved CT scan that was used for the registration.

%%%%%%%%%%%%%%%%%%%%%%%%%%%%%%%%%%%%%%%%%%%%%%%%%%%%%%%%%%%%%%%%%%%%%%%%%%%%%%%%
\begin{table}[h!]
\centering
\resizebox{\textwidth}{!}{%
\begin{tabular}{c|cccc|c}
\hline
Normalised Mutual Information & \multicolumn{4}{|c|}{Model Version} & Baseline \\ \hline
\multirow{3}{*}{Test Set parameters} & \multicolumn{4}{|c|}{Rigid} & Rigid \\ \cline{2-6}
      & \multicolumn{2}{c}{No Shear} & \multicolumn{2}{c|}{Shear$\pm0.1$} \\\cline{2-5}
      & $\theta\pm0.2$, $T\pm 20$ & $\theta\pm0.4$, $T\pm 40$ & $\theta\pm0.2$, $T\pm 20$ & $\theta\pm0.4$, $T\pm 40$ \\\cline{2-5}
      $\theta \pm 0.2$, $T \pm 20$, No Shear & 0.2462 & 0.2464 & 0.2434 & 0.2426 & 0.2182 \\
      $\theta \pm 0.4$, $T \pm 40$, No Shear & 0.2430 & \textbf{0.2451} & 0.2349 & 0.2369 & 0.2137 \\\cline{2-6}
      & \multicolumn{4}{|c|}{Affine} & Affine\\ \cline{2-6}
      $\theta \pm 0.2$, $T \pm 20$, No Shear & 0.2475 & 0.2477 & 0.2476 & \textbf{0.2478} & 0.2055\\
      $\theta \pm 0.4$, $T \pm 40$, No Shear & 0.2441 & 0.2442 & 0.2444 & 0.2444 & 0.2017\\\hline
\end{tabular}%
}
\caption{CT-MRI registration results showing the MI score, for models trained on the different datasets. Rightmost column is MRI-MRI registration with baseline model for comparison. Bolded numbers denote the model with best performance for the relevant test set. Note that each test set here is spread out over two rows.}
\label{table:tuned_no_shear_MI_results}
\end{table}
%%%%%%%%%%%%%%%%%%%%%%%%%%%%%%%%%%%%%%%%%%%%%%%%%%%%%%%%%%%%%%%%%%%%%%%%%%%%%%%%

Examples showing the registration process can be seen in Figure \ref{fig:reg_example}. This is included to show that despite the NMI scores being in the range of roughly $0.20-0.22$, the resulting registration is visually very close, albeit with small differences. These differences can be visualised better in Figure \ref{fig:reg_diff}, which shows the absolute voxel errors of a single slice.

\begin{figure}[h!]
\centering
\includegraphics[width=0.90\textwidth]{images/registration_example_ct_mr.png}
\\
\includegraphics[width=0.90\textwidth]{images/registration_example_mr_mr.png}
\caption{Examples showing a 2D slice of a registered image. From left to right: moving, fixed, moved MRI, moved CT. The moving images have been subject to the exact same transform, for ease of comparison. The moved images in the top row have been moved according to the affine transform found by registering the moving CT to the fixed MRI. The bottom row is transformed according to the moving MRI.}\label{fig:reg_example}
\end{figure}

%Best model by dice: (4, (0.20270081025856937, 20.503577818261668))
% (0.4, 40, None, 'rigid') model
Our use of the SynthSeg for segmentation outputs a volume with 32 different classes. This is used as a tool to evaluate the clinical viability of the registration. An example showing a slice of segmented brains can be seen in Figure \ref{fig:seg_diff}.

\begin{figure}[h!]
  \centering
  \begin{minipage}{0.45\textwidth}
    \centering
    \includegraphics[width=0.48\textwidth]{images/registration_example_ct_mr_diff.png}
    \includegraphics[width=0.48\textwidth]{images/registration_example_mr_mr_diff.png}
    \caption{Example showing showing the same 2D slice as in Figure \ref{fig:reg_example}. Left image is CT-MRI registered, right is MRI-MRI registered. This shows the absolute differences between the fixed and moved MRI images.}\label{fig:reg_diff}
  \end{minipage}\hfill
  \begin{minipage}{0.45\textwidth}
    \centering
    \includegraphics[width=\textwidth]{images/segm_fixed_moved.png}
    \caption{Example showing a 2D slice of the segmentation. Left is ``ground truth'' segmentation of the fixed volume. Right is segmentation of moved image.}\label{fig:seg_diff}
  \end{minipage}
\end{figure}

\begin{figure}[h!]
    \centering
    \includegraphics[width=0.48\textwidth]{images/dice/data0.2_20_None_model0.4_40_0.1_rigid.png}
    \includegraphics[width=0.48\textwidth]{images/dice/data0.4_40_None_model0.4_40_None_rigid.png}
    \caption{Dice scores and Hausdorff distances of the SynthSeg segmentations of moved images using one of the best performing registration models.}\label{fig:dice_haus_example}
\end{figure}
 % unfinished
%\subsection{Data} % Probably put these in results .tex file

\section{Discussion}
In this paper, we have successfully demonstrated the capability of the Synthmorph network to register not only various MRI modalities to each other, but with extra training, also CT scans. However, we have been unable to reproduce their success in clinically relevant registration. Specifically, their paper reports a Dice score of over $0.6$ on white matter, grey matter and cerebrospinal fluid across various forms of multi-modal registration.\cite{synthmorph} In our experiments however, we were only ever able to achieve a mean Dice score of $0.3312$ with their model, even after fine-tuning on the SynthRAD data.

This is likely due to the fact that they calculate this metric differently than in this paper. In their paper, they use an overlapping, one-hot encoded segmentation, meaning that any one voxel can be segmented as multiple classes. This makes any small mistakes that could otherwise potentially significantly decrease the score less punishing. The formula they use to calculate their ``soft'' Dice can be seen in Equation \ref{eq:synth_dice}\cite{synthmorph}, formulated as a loss function. In this paper, we use a non-overlapping segmentation of the volumes, and calculate the Dice score using the formula shown in Equation \ref{eq:dice}.

\begin{align}
\mathcal{L}(T_\theta, s_m, s_f) = -\frac{2}{|J|} \sum_{j \in J} \sum_{x \in \Omega} \frac{ \left( s_{m}|_j \circ T_\theta (x) \times s_{f}|_j(x) \right) }{ \left( s_{m}|_j \circ T_\theta (x) + s_{f}|_j(x) \right) }\label{eq:synth_dice}
\end{align}

However, the poor overall segmentation performance shown in Figure \ref{fig:dice_haus_example} would seem to indicate that the overall performance is still not all that good, with mean Hausdorff distances for nearly all classes being at or above $20$.

%In this paper, we successfully created a method of automatically registering CT scans to T1 MRI scans, demonstrating that the multi-modal capabilities of the Synthmorph network can be extended beyond the various modalities of MRI scans to CT scans as well. This extra capability also did not seem to come at the expense of poorer performance in the original T1 modality,

% though it has not been explored further to show whether this holds for different MRI modalities, or data

%Best model White matter dice: 0.3312

% TALK ABOUT PERFORMANCE


\subsection{Limitations}
This paper has demonstrated that Synthmorph can achieve roughly the same performance on CT-MRI registration as it does with just MRI-MRI registration. However, this is not a guarantee that the model can generalise to other, different types of scans such as PET scans, without further training. This has not been tested, but given the models baseline performance on CT-MRI registration, it is safe to assume.
 % finished

\section{Future Work}
\subsection{Validation of model multi-modal capabilities}
The dataset in this paper consists only of T1 MRI scans, and while this is one of the modalities used in the original Synthmorph paper, we have not shown that the new capabilities demonstrated here also extend to the various other modalities of MRI scans. Evaluating this would require finding a new dataset. Given the capabilities of Synthmorph already, it would not come as a surprise that this extension will generalise.

\subsection{Expansion of model capabilities}
Given that this paper has shown the potential for the Synthmorph network to be trained to work in more than just the original MRI, further investigation should explore whether this capability can be extended to more types of scans.

\subsection{CT-MRI transformation}
Having achieved successful cross-domain registration, the logical next step would be to try and convert between CT and MRI scans. This has been demonstrated previously to be possible by Lyu \& Wang 2022\cite{diffusion_transformation}. However, methods based on diffusion have been shown to sometimes exhibit the phenomenon known as ``hallucinations''\cite{diffusion_hallucination}, where the model seems to make up things that look plausible, but have no basis in reality. This is of course an issue when dealing with medical images that could be used for diagnosis and treatment of patients. If a diffusion model hallucinates when generating an image in this context, it could lead to misdiagnoses and incorrect treatments. Creating a network that doesn't hallucinate would be a big step forward in that context.
 % finished?

\section{Conclusion}
In this paper, we successfully created a method of automatically registering CT scans to T1 MRI scans, demonstrating that the multi-modal capabilities of the Synthmorph network can be extended beyond the various modalities of MRI scans to CT scans as well. Through various evaluation methods, we have also shown that this extra capability did not seem to come at the expense of poorer MRI to MRI registration performance.

However, we have also demonstrated that the Synthmorph network itself it likely not yet good enough to be of use in clinical settings in practice.
 % unfinished



%footnote example: \footnote{Text here will appear in footnote}


% Neat bibliography
% Feel free to uncomment if you refer to books, articles etc.
\bibliographystyle{unsrtnat}
\bibliography{references}

\newpage
\section{Appendix}
\subsection{All Dice scores and Hausdorff distances}
\begin{figure}
  \centering
  \includegraphics[width=0.48\textwidth]{images/dice/data0.2_20_None_model0.2_20_None_rigid.png}
  \includegraphics[width=0.48\textwidth]{images/dice/data0.2_20_None_model0.2_20_None_affine.png}
  \\
  \includegraphics[width=0.48\textwidth]{images/dice/data0.2_20_None_model0.2_20_0.1_rigid.png}
  \includegraphics[width=0.48\textwidth]{images/dice/data0.2_20_None_model0.2_20_0.1_affine.png}
  \\
  \includegraphics[width=0.48\textwidth]{images/dice/data0.2_20_None_model0.4_40_None_rigid.png}
  \includegraphics[width=0.48\textwidth]{images/dice/data0.2_20_None_model0.4_40_None_affine.png}
  \\
  \includegraphics[width=0.48\textwidth]{images/dice/data0.2_20_None_model0.4_40_0.1_rigid.png}
  \includegraphics[width=0.48\textwidth]{images/dice/data0.2_20_None_model0.4_40_0.1_affine.png}
  \caption{Dice scores and Hausdorff distances of the SynthSeg segmentations on test sets with parameters $\theta\pm0.2$ and T$\pm20$.}
  \label{fig:dice_haus_2}
\end{figure}

\begin{figure}
  \centering
  \includegraphics[width=0.48\textwidth]{images/dice/data0.4_40_None_model0.2_20_None_rigid.png}
  \includegraphics[width=0.48\textwidth]{images/dice/data0.4_40_None_model0.2_20_None_affine.png}
  \\
  \includegraphics[width=0.48\textwidth]{images/dice/data0.4_40_None_model0.2_20_0.1_rigid.png}
  \includegraphics[width=0.48\textwidth]{images/dice/data0.4_40_None_model0.2_20_0.1_affine.png}
  \\
  \includegraphics[width=0.48\textwidth]{images/dice/data0.4_40_None_model0.4_40_None_rigid.png}
  \includegraphics[width=0.48\textwidth]{images/dice/data0.4_40_None_model0.4_40_None_affine.png}
  \\
  \includegraphics[width=0.48\textwidth]{images/dice/data0.4_40_None_model0.4_40_0.1_rigid.png}
  \includegraphics[width=0.48\textwidth]{images/dice/data0.4_40_None_model0.4_40_0.1_affine.png}
  \caption{Dice scores and Hausdorff distances of the SynthSeg segmentations on test sets with parameters $\theta\pm0.4$ and T$\pm40$. An error in the SynthSeg model causes some of these to output a Hausdorff distance of $0$.}
  \label{fig:dice_haus_4}
\end{figure}


%\subsection{Affine matrix errors mean$\pm$standard deviation}
%\begin{align*}
\text{Data: }\theta\pm0.2 T\pm20 S\pm None, \text{Model: }\theta\pm0.2 T\pm20 S\pm None \text{ (Rigid)} \\
\begin{pmatrix}
-0.0034 \pm 0.0106 &  0.0014 \pm 0.1426 &  0.0022 \pm 0.1600 & -0.0911 \pm 34.4670 \\
-0.0003 \pm 0.1421 & -0.0031 \pm 0.0104 &  0.0017 \pm 0.1609 &  0.0225 \pm 34.6406 \\
-0.0021 \pm 0.1618 & -0.0018 \pm 0.1597 & -0.0031 \pm 0.0098 &  0.9944 \pm 30.0889 \\
 - &  - &  - &  - \\
\end{pmatrix}
\end{align*}

\begin{align*}
\text{Data: }\theta\pm0.2 T\pm20 S\pm None, \text{Model: }\theta\pm0.2 T\pm20 S\pm None \text{ (Affine)} \\
\begin{pmatrix}
 0.0066 \pm 0.0144 & -0.0042 \pm 0.1491 & -0.0030 \pm 0.1632 & -0.1911 \pm 34.4921 \\
-0.0016 \pm 0.1431 &  0.0043 \pm 0.0125 & -0.0001 \pm 0.1608 & -0.4891 \pm 34.7599 \\
-0.0027 \pm 0.1642 & -0.0012 \pm 0.1665 &  0.0039 \pm 0.0127 & -0.1145 \pm 30.1728 \\
 - &  - &  - &  - \\
\end{pmatrix}
\end{align*}

\begin{align*}
\text{Data: }\theta\pm0.2 T\pm20 S\pm None, \text{Model: }\theta\pm0.2 T\pm20 S\pm0.1 \text{ (Rigid)} \\
\begin{pmatrix}
-0.0054 \pm 0.0101 &  0.0006 \pm 0.1379 &  0.0007 \pm 0.1512 &  0.3159 \pm 34.0235 \\
 0.0016 \pm 0.1376 & -0.0049 \pm 0.0099 & -0.0014 \pm 0.1539 & -0.1851 \pm 34.3785 \\
 0.0002 \pm 0.1526 &  0.0013 \pm 0.1529 & -0.0053 \pm 0.0094 & -0.1029 \pm 29.4971 \\
 - &  - &  - &  - \\
\end{pmatrix}
\end{align*}

\begin{align*}
\text{Data: }\theta\pm0.2 T\pm20 S\pm None, \text{Model: }\theta\pm0.2 T\pm20 S\pm0.1 \text{ (Affine)} \\
\begin{pmatrix}
 0.0052 \pm 0.0141 & -0.0055 \pm 0.1510 & -0.0009 \pm 0.1624 & -0.1994 \pm 34.3003 \\
-0.0016 \pm 0.1433 &  0.0025 \pm 0.0122 & -0.0008 \pm 0.1604 & -0.1589 \pm 34.6442 \\
-0.0011 \pm 0.1630 & -0.0009 \pm 0.1656 &  0.0047 \pm 0.0124 & -0.5276 \pm 29.8509 \\
 - &  - &  - &  - \\
\end{pmatrix}
\end{align*}

\begin{align*}
\text{Data: }\theta\pm0.2 T\pm20 S\pm None, \text{Model: }\theta\pm0.4 T\pm40 S\pm None \text{ (Rigid)} \\
\begin{pmatrix}
-0.0020 \pm 0.0113 & -0.0019 \pm 0.1469 & -0.0006 \pm 0.1648 &  0.4640 \pm 34.6050 \\
 0.0022 \pm 0.1459 & -0.0017 \pm 0.0111 &  0.0004 \pm 0.1658 & -0.2540 \pm 35.0319 \\
 0.0005 \pm 0.1670 & -0.0010 \pm 0.1643 & -0.0015 \pm 0.0104 &  0.2362 \pm 30.4484 \\
 - &  - &  - &  - \\
\end{pmatrix}
\end{align*}

\begin{align*}
\text{Data: }\theta\pm0.2 T\pm20 S\pm None, \text{Model: }\theta\pm0.4 T\pm40 S\pm None \text{ (Affine)} \\
\begin{pmatrix}
 0.0010 \pm 0.0131 & -0.0021 \pm 0.1514 & -0.0028 \pm 0.1665 &  0.3269 \pm 34.7316 \\
-0.0022 \pm 0.1446 &  0.0045 \pm 0.0128 & -0.0003 \pm 0.1626 & -0.3532 \pm 34.6806 \\
 0.0003 \pm 0.1670 & -0.0014 \pm 0.1673 &  0.0043 \pm 0.0126 & -0.5906 \pm 30.3261 \\
 - &  - &  - &  - \\
\end{pmatrix}
\end{align*}

\begin{align*}
\text{Data: }\theta\pm0.2 T\pm20 S\pm None, \text{Model: }\theta\pm0.4 T\pm40 S\pm0.1 \text{ (Rigid)} \\
\begin{pmatrix}
-0.0035 \pm 0.0108 & -0.0025 \pm 0.1421 &  0.0008 \pm 0.1590 &  0.5536 \pm 34.7474 \\
 0.0041 \pm 0.1416 & -0.0036 \pm 0.0104 & -0.0039 \pm 0.1579 & -0.0128 \pm 34.7786 \\
-0.0004 \pm 0.1606 &  0.0034 \pm 0.1566 & -0.0036 \pm 0.0098 & -0.0342 \pm 30.4028 \\
 - &  - &  - &  - \\
\end{pmatrix}
\end{align*}

\begin{align*}
\text{Data: }\theta\pm0.2 T\pm20 S\pm None, \text{Model: }\theta\pm0.4 T\pm40 S\pm0.1 \text{ (Affine)} \\
\begin{pmatrix}
 0.0029 \pm 0.0136 & -0.0022 \pm 0.1524 & -0.0017 \pm 0.1648 & -0.1812 \pm 34.9443 \\
-0.0027 \pm 0.1448 &  0.0051 \pm 0.0133 & -0.0005 \pm 0.1629 & -0.3127 \pm 35.1531 \\
-0.0018 \pm 0.1670 & -0.0024 \pm 0.1702 &  0.0045 \pm 0.0125 & -0.1570 \pm 30.5087 \\
 - &  - &  - &  - \\
\end{pmatrix}
\end{align*}

\begin{align*}
\text{Data: }\theta\pm0.4 T\pm40 S\pm None, \text{Model: }\theta\pm0.2 T\pm20 S\pm None \text{ (Rigid)} \\
\begin{pmatrix}
-0.0151 \pm 0.0437 & -0.0066 \pm 0.2850 &  0.0002 \pm 0.3099 &  2.2282 \pm 67.1143 \\
 0.0070 \pm 0.2818 & -0.0128 \pm 0.0375 &  0.0030 \pm 0.3023 &  0.1143 \pm 69.1888 \\
-0.0018 \pm 0.3147 & -0.0035 \pm 0.2875 & -0.0135 \pm 0.0415 &  2.1211 \pm 67.1899 \\
 - &  - &  - & - \\
\end{pmatrix}
\end{align*}

\begin{align*}
\text{Data: }\theta\pm0.4 T\pm40 S\pm None, \text{Model: }\theta\pm0.2 T\pm20 S\pm None \text{ (Affine)} \\
\begin{pmatrix}
-0.0010 \pm 0.0464 & -0.0077 \pm 0.2972 & -0.0036 \pm 0.3163 &  0.8346 \pm 67.4163 \\
 0.0043 \pm 0.2859 &  0.0034 \pm 0.0419 & -0.0008 \pm 0.3020 & -0.7994 \pm 69.2903 \\
-0.0027 \pm 0.3202 & -0.0038 \pm 0.2991 &  0.0022 \pm 0.0471 & -0.0250 \pm 67.3909 \\
 - &  - &  - & - \\
\end{pmatrix}
\end{align*}

\begin{align*}
\text{Data: }\theta\pm0.4 T\pm40 S\pm None, \text{Model: }\theta\pm0.2 T\pm20 S\pm0.1 \text{ (Rigid)} \\
\begin{pmatrix}
-0.0225 \pm 0.0410 & -0.0100 \pm 0.2763 &  0.0014 \pm 0.2938 &  3.2364 \pm 65.6569 \\
 0.0122 \pm 0.2733 & -0.0190 \pm 0.0361 & -0.0059 \pm 0.2910 &  0.9265 \pm 68.8837 \\
-0.0006 \pm 0.2986 &  0.0048 \pm 0.2774 & -0.0221 \pm 0.0391 &  1.0393 \pm 65.1856 \\
 - &  - &  - & - \\
\end{pmatrix}
\end{align*}

\begin{align*}
\text{Data: }\theta\pm0.4 T\pm40 S\pm None, \text{Model: }\theta\pm0.2 T\pm20 S\pm0.1 \text{ (Affine)} \\
\begin{pmatrix}
-0.0035 \pm 0.0464 & -0.0081 \pm 0.2997 & -0.0019 \pm 0.3151 &  0.9735 \pm 66.8732 \\
 0.0043 \pm 0.2854 &  0.0018 \pm 0.0418 & -0.0022 \pm 0.3000 & -0.3630 \pm 69.0953 \\
-0.0005 \pm 0.3187 & -0.0030 \pm 0.2982 &  0.0016 \pm 0.0463 & -0.4109 \pm 66.7411 \\
 - &  - &  - & - \\
\end{pmatrix}
\end{align*}

\begin{align*}
\text{Data: }\theta\pm0.4 T\pm40 S\pm None, \text{Model: }\theta\pm0.4 T\pm40 S\pm None \text{ (Rigid)} \\
\begin{pmatrix}
-0.0094 \pm 0.0463 & -0.0059 \pm 0.2924 & -0.0015 \pm 0.3192 &  1.3733 \pm 67.9006 \\
 0.0052 \pm 0.2890 & -0.0071 \pm 0.0394 & -0.0010 \pm 0.3125 &  0.3056 \pm 70.4195 \\
 0.0005 \pm 0.3246 & -0.0005 \pm 0.2966 & -0.0071 \pm 0.0439 &  0.3181 \pm 68.5773 \\
 - &  - &  - & - \\
\end{pmatrix}
\end{align*}

\begin{align*}
\text{Data: }\theta\pm0.4 T\pm40 S\pm None, \text{Model: }\theta\pm0.4 T\pm40 S\pm None \text{ (Affine)} \\
\begin{pmatrix}
-0.0022 \pm 0.0481 & -0.0052 \pm 0.3013 & -0.0049 \pm 0.3233 &  0.8341 \pm 67.9933 \\
 0.0018 \pm 0.2893 &  0.0050 \pm 0.0426 & -0.0007 \pm 0.3055 & -0.5784 \pm 69.6359 \\
 0.0010 \pm 0.3269 & -0.0034 \pm 0.3015 &  0.0050 \pm 0.0485 & -1.0711 \pm 67.9044 \\
 - &  - &  - & - \\
\end{pmatrix}
\end{align*}

\begin{align*}
\text{Data: }\theta\pm0.4 T\pm40 S\pm None, \text{Model: }\theta\pm0.4 T\pm40 S\pm0.1 \text{ (Rigid)} \\
\begin{pmatrix}
-0.0150 \pm 0.0439 & -0.0088 \pm 0.2853 &  0.0006 \pm 0.3089 &  2.1649 \pm 67.9681 \\
 0.0109 \pm 0.2820 & -0.0146 \pm 0.0373 & -0.0072 \pm 0.2963 &  0.9738 \pm 69.9987 \\
-0.0003 \pm 0.3140 &  0.0052 \pm 0.2813 & -0.0158 \pm 0.0407 &  0.6125 \pm 67.2568 \\
 - &  - &  - & - \\
\end{pmatrix}
\end{align*}

\begin{align*}
\text{Data: }\theta\pm0.4 T\pm40 S\pm None, \text{Model: }\theta\pm0.4 T\pm40 S\pm0.1 \text{ (Affine)} \\
\begin{pmatrix}
-0.0004 \pm 0.0480 & -0.0031 \pm 0.3050 & -0.0015 \pm 0.3201 & -0.3247 \pm 68.3898 \\
 0.0026 \pm 0.2903 &  0.0058 \pm 0.0434 & -0.0024 \pm 0.3043 & -0.5671 \pm 70.0385 \\
-0.0025 \pm 0.3267 & -0.0035 \pm 0.3044 &  0.0049 \pm 0.0474 & -0.5397 \pm 67.8572 \\
 - &  - &  - & - \\
\end{pmatrix}
\end{align*}


\subsection{Synthmorph baselines, with centered fixed image}
\begin{table}[h!]
\centering
\begin{minipage}{0.48\textwidth}
\centering
\resizebox{\textwidth}{!}{%
\begin{tabular}{|c|c|c|}
\hline
Generation Parameters & $\|A\hat{B} - I\|$ & $\|B - \hat{B}\|$ \\
\hline
No Rotation, No Translation, No Shear & $0$ & $0$ \\
\hline
$\theta \pm 0.4$, $T \pm 40$, No Shear & 9.2585 & 9.2585 \\
\hline
$\theta \pm 0.4$, $T \pm 40$, \text{Shear} $= \pm 0.1$ & 9.8315 & 12.3868 \\
\hline
Normalised, $\theta \pm 0.4$, $T \pm 40$, No Shear & 7.4767 & 7.4767 \\
\hline
Normalised, $\theta \pm 0.4$, $T \pm 40$, \text{Shear} $= \pm 0.1$ & 7.4014 & 10.3401 \\
\hline
$\theta \pm 0.2$, $T \pm 20$, No Shear & 5.8611 & 5.8611 \\
\hline
$\theta \pm 0.2$, $T \pm 20$, \text{Shear} $= \pm 0.1$ & 6.1733 & 7.3278 \\
\hline
Normalised, $\theta \pm 0.2$, $T \pm 20$, No Shear & 4.1896 & 4.1896 \\
\hline
Normalised, $\theta \pm 0.2$, $T \pm 20$, \text{Shear} $= \pm 0.1$ & 4.3918 & 5.6838 \\
\hline
\end{tabular}%
}
\caption{MR-MR registration results with different transformations using the baseline affine SynthMorph model.}
\label{appendix:mr_mr_results_affine}
\end{minipage}
\hfill
\begin{minipage}{0.48\textwidth}
\centering
\resizebox{\textwidth}{!}{%
\begin{tabular}{|c|c|c|}
\hline
Generation Parameters & $\|A\hat{B} - I\|$ & $\|B - \hat{B}\|$ \\
\hline
No Rotation, No Translation, No Shear & 81.4608 & 81.4608 \\
\hline
$\theta \pm 0.4$, $T \pm 40$, No Shear & 66.2298 & 66.2298 \\
\hline
$\theta \pm 0.4$, $T \pm 40$, \text{Shear} $= \pm 0.1$ & 90.0000 & 90.9563 \\
\hline
Normalised, $\theta \pm 0.4$, $T \pm 40$, No Shear & 44.1781 & 44.1781 \\
\hline
Normalised, $\theta \pm 0.4$, $T \pm 40$, \text{Shear} $= \pm 0.1$ & 45.2697 & 45.9374 \\
\hline
$\theta \pm 0.2$, $T \pm 20$, No Shear & 87.5443 & 87.5443 \\
\hline
$\theta \pm 0.2$, $T \pm 20$, \text{Shear} $= \pm 0.1$ & 269.8159 & 276.1376 \\
\hline
Normalised, $\theta \pm 0.2$, $T \pm 20$, No Shear & 42.0669 & 42.0669 \\
\hline
Normalised, $\theta \pm 0.2$, $T \pm 20$, \text{Shear} $= \pm 0.1$ & 43.9500 & 43.9333 \\
\hline
\end{tabular}%
}
\caption{CT-MR registration results with different transformations using the baseline affine SynthMorph model.}
\label{appendix:ct_mr_results_affine}
\end{minipage}
\end{table}

\begin{table}[h!]
\centering
\begin{minipage}{0.48\textwidth}
\centering
\resizebox{\textwidth}{!}{%
\begin{tabular}{|c|c|c|}
\hline
Generation Parameters & $\|A\hat{B} - I\|$ & $\|B - \hat{B}\|$ \\
\hline
No Rotation, No Translation, No Shear & $0$ & $0$ \\
\hline
$\theta \pm 0.4$, $T \pm 40$, No Shear & 7.5015 & 7.5015 \\
\hline
$\theta \pm 0.4$, $T \pm 40$, \text{Shear} $= \pm 0.1$ & 17.7726 & 18.3158 \\
\hline
Normalised, $\theta \pm 0.4$, $T \pm 40$, No Shear & 8.8842 & 8.8842 \\
\hline
Normalised, $\theta \pm 0.4$, $T \pm 40$, \text{Shear} $= \pm 0.1$ & 19.0528 & 19.1625 \\
\hline
$\theta \pm 0.2$, $T \pm 20$, No Shear & 4.5461 & 4.5461 \\
\hline
$\theta \pm 0.2$, $T \pm 20$, \text{Shear} $= \pm 0.1$ & 17.1753 & 17.1819 \\
\hline
Normalised, $\theta \pm 0.2$, $T \pm 20$, No Shear & 4.6545 & 4.6545 \\
\hline
Normalised, $\theta \pm 0.2$, $T \pm 20$, \text{Shear} $= \pm 0.1$ & 17.3669 & 17.2396 \\
\hline
\end{tabular}%
}
\caption{MR-MR registration results with different transformations using the baseline rigid SynthMorph model.}
\label{appendix:mr_mr_results_rigid}
\end{minipage}
\hfill
\begin{minipage}{0.48\textwidth}
\centering
\resizebox{\textwidth}{!}{%
\begin{tabular}{|c|c|c|}
\hline
Generation Parameters & $\|A\hat{A}^{-1} - I\|$ & $\|B - \hat{B}\|$ \\
\hline
No Rotation, No Translation, No Shear & 60.5308 & 60.5308 \\
\hline
$\theta \pm 0.4$, $T \pm 40$, No Shear & 54.8180 & 54.8180 \\
\hline
$\theta \pm 0.4$, $T \pm 40$, \text{Shear} $= \pm 0.1$ & 61.0844 & 62.0519 \\
\hline
Normalised, $\theta \pm 0.4$, $T \pm 40$, No Shear & 24.9550 & 24.9550 \\
\hline
Normalised, $\theta \pm 0.4$, $T \pm 40$, \text{Shear} $= \pm 0.1$ & 32.4703 & 33.4180 \\
\hline
$\theta \pm 0.2$, $T \pm 20$, No Shear & 50.7994 & 50.7994 \\
\hline
$\theta \pm 0.2$, $T \pm 20$, \text{Shear} $= \pm 0.1$ & 56.2187 & 57.1268 \\
\hline
Normalised, $\theta \pm 0.2$, $T \pm 20$, No Shear & 23.7351 & 23.7351 \\
\hline
Normalised, $\theta \pm 0.2$, $T \pm 20$, \text{Shear} $= \pm 0.1$ & 31.8721 & 32.9091 \\
\hline
\end{tabular}%
}
\caption{CT-MR registration results with different transformations using the baseline rigid SynthMorph model.}
\label{appendix:ct_mr_results_rigid}
\end{minipage}
\end{table}

\end{document}
